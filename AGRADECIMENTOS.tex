\chapter*{Agradecimentos}

%\textbf{(entrar no final do livro)}

Este livro é uma versão modificada de minha tese de doutorado,
intitulada \emph{Karawara: a caça e o mundo dos Awá-Guajá}, defendida em
2011 no Programa de Pós-Graduação em Antropologia Social da Universidade
de São Paulo (PPGAS-USP), e muitos são os responsáveis pela sua
concretização.

Agradeço a Fundação de Amparo à Pesquisa do Estado de São Paulo
(FAPESP), pelas bolsas de doutorado, pós-doutorado e o auxílio à
publicação que possibilitaram a realização desta obra.

Ao PPGAS-USP onde cursei mestrado e doutorado. Ao Departamento de
Antropologia da UNICAMP que me recebeu durante o pós-doutorado. E ao
Departamento de Ciências Sociais da UNIFESP, onde atualmente trabalho e
encontrei um ambiente excelente para ensino e pesquisa.

A Luísa Valentini e a editora Hedra, em especial a Jorge Sallum, pela
extrema confiança e pelo convite para que eu publicasse.

À minha família, Ana Maria, Januário, Aruan, Tainá e Raoni que sempre me
apoiaram em tudo.

Ao longo do tempo, professores, colegas, funcionários e amigos
contribuíram, cada qual de uma forma, para realização deste livro. Como
são muitos nomes, sem intenção de hierarquizar as contribuições,
coloco-os aqui em ordem alfabética. Muito obrigado a Ana Cláudia
Marques, Aristóteles Barcelos Neto, Beatriz Perrone-Moisés, Beto
Ricardo, Bruce Albert, Carlos Jardim, Carlos Travassos, Ciça Avenci
Bueno, Clarisse Jabur, Dominique Gallois, Eduardo Viveiros de Castro,
Felipe Sussekind, Felipe Vander Velden, Gabriel Barbosa, Guilherme
Orlandini Heurich, Igor Scaramuzzi, Joana Cabral, Jorge Villela, Júlia
Sauma, Juliana Rosalen, Leandro Mahalen, Luis Roberto de Paula, Madalena
Borges, Maíra Bühler, Majoí Gongora, Manuela Carneiro da Cunha, Marcelo
Yokoi, Marcos Rufino, Maria Fernanda Silva Jardim, Marina Vanzolini,
Marta Amoroso, Mauro Almeida, Miguel Palmeira, Orlando Calheiros, Paride
Bolletin, Pedro Augusto Lolli, Renato Sztutman, Rosana Diniz, Rogério do
Pateo, Salvador Schavelzon, Valéria Macedo, Vânia Maria Ferro, Vicente
Sampaio, Vinícius Jatobá e Ypuan Garcia.

Abro uma exceção agradecendo alguns nomes pontualmente.

A Fany Ricardo, Tiago Moreira dos Santos e ao ISA pelos mapas que
aparecem no livro.

A Domenico Pugliese e Sebastião Salgado pelas fotos.

No Maranhão, agradeço a Patriolino Garreto, Riba Rocha, Maria Dalva, Zé
Almir (Sesai), Antônio (Sesai), Gaúcho da Funai, Sr. João Cantú e Dona
Sueli (e toda sua família que me recebeu em Roça Grande e em São Luís),
e toda a equipe da Frente de Proteção Etno-Ambiental Awá-Guajá.

No Rio Pindaré, em Alto Alegre, agradeço a Júnior (da Funai) e sua mãe
D. Josa. No rio Caru, um agradecimento especial ao Seu Cajú e toda a sua
família, em especial ao seu filho Chico, que me deu muitas caronas de
barco. Em São João do Caru agradeço ao Sr. Zé Belo e família pelo
suporte permanente em seu hotel. E a Agostinho de Carvalho, profundo
conhecedor das artes da floresta, muito obrigado por todos os
ensinamentos dentro e fora da mata.

A Ana Letícia Silva, que caminhou junto comigo durante tantos anos, esse
livro também é seu.

A Marcio Silva, meu orientador de mestrado e doutorado, e hoje amigo,
com quem aprendi muito. A Vanessa Lea que me recebeu na Unicamp, e foi
uma interlocutora interessada e atenciosa.

A Tânia Stolze e Marcio Goldman, mestres e amigos, que há tempos atrás,
em um Rio de Janeiro distante no passado, me incentivaram a vir para São
Paulo, e durante todos esses anos foram os meus interlocutores mais
próximos.

A Marina Magalhães, grande estudiosa da língua Guajá, a quem recorri por
e-mail diversas vezes para tirar dúvidas, além de me ceder narrativas
inéditas traduzidas, que muito enriqueceram este trabalho. Sem sua
pesquisa e seu auxílio o trabalho não seria o mesmo. A Louis Carlos
Forline, que me levou para os Guajá e me apresentou à comunidade Juriti.

A Fabiana Maizza, que nos últimos anos foi a pessoa com quem mais
troquei ideias e \emph{insights} a respeito deste livro, e que de muitas
maneiras, mesmo sem perceber, me ajudou a escreve-lo por inteiro.

A Rita e Tomé.

Por fim, eu não saberia como agradecer a cada pessoa de cada comunidade
em que trabalhei. Fruto desse meu encontro com os Guajá, este livro não
escapará da parcialidade típica das etnografias, e algumas vozes - além
da minha - aparecerão mais do que outras. Tenho muito a agradecer à
aldeia Juriti, onde tive tantos aprendizados, e todos me receberam de
maneira muito generosa, abrindo suas casas e suas vidas. Apesar deste
livro ser dedicado aos Guajá, dedico-o em especial a um grupo de jovens,
filhos de muitos amigos, que eram crianças quando cheguei pela primeira
vez naquela aldeia. Hoje adultos, eles mesmos se tornaram grandes amigos
me tornando ``avô'' de seus filhos. Não sei por que, alimento a ideia de
que foram eles que, com carisma, ``amoleceram'' todos os outros que me
receberem tão bem. Obrigado a vocês Aparanỹa, Takwaria, Jui'ia,
Aparana'ia, Kawui'ia, Kiripi'ia, Jumã'ã, Panãpinuhũ e Iauxa'a.

Nas aldeias Tiracambu e Awá, onde me instalei por mais tempo durante o
pós-doutorado agradeço a Xiparẽxa'á, Wa'amãxũa, Hajmakoma'á, Makaritỹa,
Kamajrua, Ximirapia, Maihuxa'á, Pakawãja, Karapirua, Takwariroa,
Manimya, Mihaxa'a, Amiria, Jakara'yra, Makarikarỹa, Tatuxa'á, Parapijỹ,
Hajkaramỹkỹ, Arakari'ĩa, Xiwikwajỹa, Irakatakoa, Aparikya, Maijhuxa'á,
Takamỹxa'a, Piranẽxa'a, Itaxĩ, Amiria, Petu e Warixa'á.

Sem a amizade e a confiança de cada um de vocês esse livro não seria
possível!
