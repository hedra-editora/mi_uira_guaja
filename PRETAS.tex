\textbf{Uirá Garcia} é antropólogo e professor da Universidade Federal de São Paulo \textsc{(unifesp)}, com mestrado e doutorado em Antropologia Social pela Universidade de São Paulo \textsc{(usp)} e pós-doutorado na Universidade de Campinas \textsc{(unicamp)}. 
É membro do Centro de Estudos Ameríndios (\textsc{ce}st\textsc{a}) da \textsc{(usp)}, e do Núcleo de Antropologia Simétrica (\textsc{na}n\textsc{s}i) do Programa de Pós-Graduação em Antropologia Social do Museu Nacional (\textsc{ufrj}). %Seu principal tema de estudo são os Awá-Guajá, com foco nas temáticas da caça, ecologia, parentesco, sistemas de conhecimento e teoria antropológica.
	
\textbf{Crônicas de caça e criação} é uma pesquisa etnográfica sobre os Awá Guajá, povo de língua tupi-guarani, que vive ao noroeste do Maranhão. Um dos últimos povos indígenas a serem contatados, é organizado principalmente por caçadores que passaram a viver em aldeias após o contato com a \textsc{funai}. As relações que os Guajá estabelecem com seu território, assim como suas concepções cartográficas; suas formas de pensar a pessoa humana; a construção dos parentescos; a caça como atividade central da vida; e a relação dos humanos com os \emph{karawara}, entidades que habitam esferas celestiais. Para apreender e transmitir tal sistema de vida, o antropólogo passou treze meses entre os Guajá, momento em que frequentou as aldeias Juriti, Tiracambu e Awá.

\textbf{Coleção Mundo Indígena} reúne materiais produzidos com pensadores de diferentes povos indígenas e pessoas que pesquisam, trabalham ou lutam pela garantia de seus direitos. Os livros foram feitos para serem utilizados pelas comunidades envolvidas na sua produção, e por isso uma parte significativa das obras é bilíngue. Esperamos divulgar a imensa diversidade linguística dos povos indígenas no Brasil, que compreende mais de 150 línguas pertencentes a mais de trinta famílias linguísticas.



