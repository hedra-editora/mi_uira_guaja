\textbf{Uirá Garcia} é antropólogo e professor da Universidade Federal de São Paulo \textsc{(unifesp)}, com mestrado e doutorado em Antropologia Social pela Universidade de São Paulo \textsc{(usp)} e pós-doutorado na Universidade de Campinas \textsc{(unicamp)}. 
%É membro do Centro de Estudos Ameríndios (\versal{CE}st\versal{A}) da \versal{(USP)} e do Núcleo de Antropologia Simétrica (\versal{NA}n\versal{S}i) do Programa de Pós-Graduação em Antropologia Social do Museu Nacional/\versal{UFRJ}. Seu principal tema de estudo são os Awá-Guajá, com foco nas temáticas da caça, ecologia, parentesco, sistemas de conhecimento e teoria antropológica.
	
\textbf{Crônicas de caça e criação} é uma pesquisa etnográfica sobre os Awá-Guajá, povo de língua Tupi-Guarani da Amazônia Oriental, precisamente do noroeste do Maranhão. Um dos últimos povos indígenas a serem contatados pelo Estado brasileiro, é constituído por caçadores habilidosos, que estruturam grande parte de seu sistema de pensamento nessa atividade e passaram a viver em aldeias após o contato iniciado pela Funai. No livro, Uirá Garcia se debruça, principalmente, sobre as relações que os Guajá estabelecem com seu território, assim como suas concepções cartográficas; suas formas de pensar a pessoa humana; a construção dos parentescos; a caça como atividade central da vida; e a relação dos humanos com os \emph{karawara}, entidades que habitam esferas celestiais. Para apreender e transmitir tal sistema de vida, o antropólogo passou treze meses entre os Guajá, momento em que frequentou as aldeias Juriti, Tiracambu e Awá.

\textbf{Mundo Indígena}, coleção da Editora Hedra, reúne, de um lado, as cosmologias, histórias
e reflexões de povos indígenas nas palavras de seus próprios pensadores e, de outro, coletâneas
e trabalhos acadêmicos de grandes estudiosos da questão indígena no Brasil, reafirmando assim sua
existência e relevância em seus próprios termos.\par


