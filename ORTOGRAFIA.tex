\hypertarget{luxedngua-e-convenuxe7uxe3o-ortogruxe1fica}{%
\section{Língua e convenção
ortográfica}\label{luxedngua-e-convenuxe7uxe3o-ortogruxe1fica}}

A língua Guajá foi inicialmente estudada por Péricles Cunha (1988) e
mais recentemente por Marina Magalhães (2007). Esta língua pertence ao
subgrupo VIII da família linguística Tupí-Guaraní, que inclui também o
Takunyapé, o Ka'apor, o Waiampi, o Wayampipukú, o Emérillon, o Amanayé,
o Anambé, o Turiwára e o Zo'é (Rodrigues, 1984/85 e Magalhães 2007). A
grafia dos termos em Guajá aqui utilizadas baseia-se fundamentalmente no
trabalho de Magalhães (2007) que realiza uma análise morfológica e
sintática dessa língua. Todas as palavras na língua Guajá, bem como em
outras línguas indígenas, estão em itálico e as traduções entre aspas.
Utilizei uma convenção fonética cujos valores dos sons aproximados são:

\textbf{Vogais}

\begin{longtable}[]{@{}ll@{}}
\toprule
/a/ & vogal central baixa (como \emph{a} em português);\tabularnewline
\midrule
\endhead
/e/ & vogal anterior média não-arredondada (como \emph{e} em
português);\tabularnewline
/i/ & vogal anterior alta não arredondada (como \emph{i} em
português);\tabularnewline
/y/ & vogal central alta não arredondada (tal como encontrada em outras
línguas Tupi);\tabularnewline
/o/ & vogal posterior média arredondada (como \emph{o} em
português);\tabularnewline
/u/ & vogal posterior alta arredondada (como \emph{u} em
português);\tabularnewline
\bottomrule
\end{longtable}

Todas as vogais podem ser nasalisadas; para tanto, utilizo o til em
todos os casos {[}ã, ẽ, ĩ, ỹ, õ, ũ{]}, e geralmente a presença de uma
vogal nasal acarreta a nasalisação das vogais e consoantes que lhes são
próximas.

\textbf{Consoantes}

\begin{longtable}[]{@{}ll@{}}
\toprule
/p/ & oclusiva bilabial surda (como \emph{p} em
português);\tabularnewline
\midrule
\endhead
/t/ & oclusiva alveolar surda (como \emph{t} em
português);\tabularnewline
/x/ & oclusiva dental palatalizada (soa como o \emph{t} de \emph{tia} no
falar carioca, porém ocorrendo diante das vogais);\tabularnewline
/k/ & oclusiva velar surda (como o \emph{c} em \emph{casa} em
português);\tabularnewline
/kw/ & oclusiva velar surda labializada (como em \emph{quarto} em
português);\tabularnewline
/m/ & nasal bilabial (como \emph{m} em português);\tabularnewline
/n/ & nasal dental (como \emph{n} em português);\tabularnewline
/j/ & aproximante palatal cujo som é o equivalente ao produzido em
ditongos com \emph{i}, como nas palavras "sai" e "meia" em português;
seguida de vogal nasal (ex.: \emph{jã `cantar')}, passa a ser
pronunciada como nasal alveopalatal sonora (como o \emph{nh} em
\emph{manhã} em português)\tabularnewline
/r/ & vibrante simples (tepe) (como \emph{para} em
português);\tabularnewline
/h/ & fricativa glotal (como \emph{heaven} em inglês);\tabularnewline
/w/ & contínua bilabial sem fricção (como em \emph{power} em
inglês);\tabularnewline
/ ' / & oclusão glotal;\tabularnewline
\bottomrule
\end{longtable}

Assim como as vogais, muitas consoantes podem ser nasalisadas, e são
aqui identificadas com um til.

A ausência de acento na ortografia da língua deve-se à previsibilidade
das sílabas tônicas já que, de maneira regular, todas as palavras são
oxítonas, isto é, terminam em sílabas tônicas, exceto as que terminam em
vogal \emph{a} depois de outra vogal ou depois das consoantes r, n e j.
Assim, \emph{awa} e \emph{mukuri} são pronunciadas como ``awá'' e
``mukurí'', mas \emph{mihua}, \emph{jakarea, Maira, amỹna e takaja} são
pronunciadas como ``mihúa'', ``jakaréa'', ``Maíra'', ``amỹna'' e
``takája''.

Quanto à grafia dos povos indígenas, acompanho a convenção da Associação
Brasileira de Antropologia, sem flexionar os nomes.
