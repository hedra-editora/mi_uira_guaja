
\chapter{Referências bibliográficas}

\begin{bibliohedra}
\tit{Albert}, Bruce. ``O ouro canibal e a queda do céu: uma crítica xamânica da
economia política da natureza''. In: \emph{Série Antropologia},
\emph{174}. Brasília, 1995.

\tit{Albert}, Bruce \& \textsc{miliken}, William. \emph{Urihi A: a terra"-floresta Yanomami}. São Paulo: \textsc{isa}"-\textsc{ird}, 2009.

\tit{Almeida}, Mauro W. B. \emph{Caipora e outros conflitos ontológicos} (mimeo).
Conferência Quartas Indomáveis, São Carlos, 2007.

\tit{AmaZone}, Projeto. \emph{A onça e a diferença}, 2010.

\tit{Andrade}, Lucia M. M. \emph{O Corpo e os Cosmos: Relações de Gênero e o
Sobrenatural entre os Asurini do Tocantins}. Dissertação de Mestrado.
São Paulo: Universidade de São Paulo, 1992.

\tit{Århem}, Kaj. \emph{La red cósmica de la alimentación}. In: \textsc{descola},
Philppe \& \textsc{pálsson}, Gísli. México: Siglo \textsc{xxi}, 2001.

\tit{Asad}, Talal. ``The Concept of Cultural Translation in British Social
Anthropology''. In: \textsc{clifford}, James \& \textsc{marcus}, George E. (eds.).
\emph{Writing Culture. The Poetics and Politics of Ethnography}.
Berkley: University of California Press, 1986.

\tit{Balée}, William. \emph{Culturas da floresta: apontamentos críticos sobre a
ecologia cultural na Amazônia}, trabalho lido no Simpósio \textsc{aba}/\textsc{anpocs}, 1987.

\titidem. ``O povo da capoeira velha: caçadores"-coletores das
terras baixas da América do Sul''. Trabalho apresentado na Conferência
Amazônica da Fundação Memorial da América Latina, 25 mar. 1992. 

\titidem. ``People of the Fallow: A historical Ecology of Foraging
in Lowland South América''. In: \textsc{redford}, Kent H \&  \textsc{padoch}, Christine
(orgs). \emph{Conservation of neotropical forests: working from
traditional resource use}. New York: Columbia University Press, 1992.

\titidem. \emph{Footprints of the forest: Ka'apor ethnobotany:
the historical ecology of plant utilization by an amazonian people}. New
York: Columbia University Press, 1994.

\titidem. ``Language, law, and land in Pre"-Amazonian Brazil''. In:
\emph{Texas International Law Journal} vol.\,32, n.\,1, 1997.

\titidem. ``The Sirionó of the Lanos de Mojos, Bolívia''. In: \textsc{daly}, Richard
\& \textsc{lee}, Richard (orgs). \emph{The Cambridge Encyclopedia of Hunters and
Gatherers}. Cambridge: Cambridge University Press, 1999.

\titidem. ``Antiquity of traditional ethnobiological knowledge in
Amazonia: the Tupí-Guaraní family and time''. In \emph{American Society
for Ethnohistory}, Tulane University, 2000.

\titidem. \emph{Cultural Forests of the Amazon. A Historical
Ecology of People and their Landscapes}. Tuscaloosa, Alabama: The
University of Alabama Press, 2013.

\tit{Barbosa}, Gabriel Coutinho. \emph{Os Aparai e Wayana e suas redes de intercâmbio}.
Tese de Doutorado. São Paulo: Universidade de São Paulo, 2007.

\tit{Bateson}, Gregory. ``Toward a Theory of Schizophrenia''. In:
\emph{Steps to na ecology of mind.} Chicago/London: The University of
Chicago Press, 1956 (2000).

\tit{Beckerman}, S. \& \textsc{valentine}, P. (eds.).
\emph{Cultures of Multiple Fathers: The theory and
practice of partible paternity in South America}. Gainesville,
University of Florida Press, 2002.

\tit{Bird"-David}, Nurit. ``The Giving Environment: Another Perspective on the
Economics System of Hunter"-gatherers''. In: \emph{Current Anthropology}, 31, 1990, p.\,183--96.

\tit{Bonilla}, Oiara. ``O bom patrão e o inimigo voraz: predação e comércio na
cosmologia paumari''. \emph{Mana. Estudos de Antropologia
Social}, 11(1), 2005, p.\,41--66.

\tit{Brightman}, Marc. ``Creativity and control: property in Guianese
Amazonia''. \emph{Journal de la Societe des Americanistes de Paris},
96(1), 2010, p.\,135--167.

\tit{Brightman}, Robert. ``The Sexual Division of Foraging Labor: Biologt, Taboo,
and Gender Politics''. \emph{Comparative Studies in Society and History},
v.\,38, n.\,4, 1996, pp. 687--729.

\tit{Brochado}, J. P. \emph{An ecological model of the pread of pottery and
agriculture into Eastern South America}, ph.d. dissertation, University
of Illinois at Urbana"-Champaign, Ann Arbor \textsc{umi}, Inc, 1984.

\tit{Cabral}, Joana. \emph{Entre plantas e palavras. Modos de constituição de
saberes entre os Wajãpi (\textsc{ap})}. Tese de Doutorado, Universidade de São
Paulo, 2012.

\tit{Calheiros}, Orlando. \emph{Aikewara: Esboços de uma sociocosmologia tupi"-guarani}. Tese de Doutorado. Rio de Janeiro: Museu Nacional/\textsc{ufrj}, 2014.

\tit{Carneiro da Cunha}, Manuela. \emph{Os mortos e os outros: uma análise do sistema
funerário e da noção de pessoa entre os índios Krahô}. São Paulo:
Hucitec, 1978.

\tit{Cesarino}, Pedro. \emph{\textsc{oniska}: A poética da morte e do mundo entre os
Marubo da Amazônia ocidental}. Tese de Doutorado. Rio de Janeiro. \textsc{ppgas},
Museu Nacional, 2008.

\titidem. ``Donos e duplos: relações de conhecimento, propriedade
e autoria entre os Marubo''. \emph{Revista de Antropologia}, 53(1), 2010, p.\,147--197.

\tit{cimi} -- Conselho Indigenista Missionário. ``Ameaçados, Awá Guajá isolados aceitam contato no Maranhão'', 2015.

\tit{Clastres}, Hélène. \emph{Terra sem mal: o profetismo tupi"-guarani}.
São Paulo: Brasiliense, 1978 (1975).

\tit{Clastres}, Pierre. ``Entre silêncio e dialogo''. In: \emph{L'arc Documentos"-
Lévi"-Strauss}. São Paulo: Editora Documentos, 1968.

\titidem. \emph{Crônica dos índios Guayaki: o que sabem os
Aché, caçadores nômades do Paraguai}. Rio de Janeiro: Editora 34, 1995 (1972).

\titidem. \emph{A sociedade contra o Estado -- pesquisas de
antropologia política}. São Paulo: Cosac \& Naify, 2003 (1974).

\titidem. ``A economia Primitiva''. In: \emph{Arqueologia da
Violência. São Paulo: Cosac \& Naify}, 2004 (1976).

\titidem{Coelho de Souza}, Marcela. ``Parentes de sangue: incesto, substância e relação no
pensamento Timbira''. In: \emph{Mana. Estudos de Antropologia Social}, 10(1), 2004, p.\,25--60.

\tit{Collier}, Jane \& \textsc{yanagisako}, Sylvia.
``Toward a Unified analysis of gender and kinship''. In:
\textsc{collier}, Jane \& \textsc{yanagisako}, Sylvia (eds.). \emph{Gender and Kinship.
Essays toward a Unified Analysis}. Standford: Standford University
Press, 1987.

\tit{Cormier}, Loretta A. \emph{The ethnoprimatology of the Guajá indians of
Maranhão}, \emph{Brazil}. Doctor Thesis. Department of Anthropology --
Tulane University, 2000.

\titidem. \emph{Kinship With Monkeys: The Guajá foragers of eastern
Amazonia}. New York: Columbia University Press, 2003.

\titidem. ``Um aroma no ar: a ecologia histórica das plantas
anti"-fantasma entre os Vuajá da Amazônia''. In: \emph{Mana.
Estudos de Antropologia Social}, 11(1), 2005, p.\,129--154.

\tit{Corsín Jimenez}, Alberto \& \textsc{willerslev}, Rane.
``An anthropological concept of the concept:
reversibility among the Siberian Yukaghirs''. \emph{Journal of the Royal
Anthropological Institute} (\textsc{n.s.}), 13, 2007, p.\,527--544.

\tit{Costa}, Luiz Costa. ``Alimentação e comensalidade entre os Kanamari da
Amazônia Ocidental''. \emph{Mana. Estudos de Antropologia Social}, 19(3), 2013, p.\,473--504.

\tit{Cunha}, Péricles. \emph{Análise fonêmica preliminar da língua Guajá}.
Dissertação de Mestrado. Universidade de Campinas, 1987.

\tit{Da Matta}, Roberto. \emph{Ensaios de Antropologia Estrutural}. Petrópolis:
Vozes, 1973.

\titidem. \emph{Um mundo dividido: a estrutura social dos índios
Apinayé}. Petrópolis, Vozes, 1976.

\tit{Dal Poz Neto}, João. \emph{Dádivas e Dívidas na Amazônia -- parentesco,
economia e ritual nos Cinta"-Larga}. Tese de Doutorado. Campinas:
Unicamp, 2004.

\tit{Descola}, Philippe. \emph{La nature domestique: symbolisme et praxis dans
l'écologie des Achuar}. Paris: Maison des Sciences de l'Homme, 1986.

\titidem. ``El determinismo raquítico''. In: \emph{Etnoecológica},
v.\,1, n.\,1, abr. 1991.

\titidem. ``Societies of nature and the nature of society''. In: \textsc{kuper}, A.
(org.). \emph{Conceptualizing society}. Londres: Routledge, 1992, p.\,107--126.

\titidem. ``Les affinités sélectives: alliance, guerre et predation
dans l'ensemble jivaro''. In: \textsc{descola}, P. \& \textsc{taylor}, A.-C. (orgs.). \emph{La remontée de l'Amazone: anthropologie et histoire des sociétes
amazoniennes. L'Hommme, 126--128}, 1993, p.\,171--90.

\titidem. \emph{La Selva Culta: simbolismo y praxis en la ecologia del
los Ashuar}. Quito: Abya"-Yala, 1996.

\titidem. ``Estrutura ou sentimento: a relação com o animal na
amazônia''. In: \emph{Mana. Estudos de Antropologia Social}, 4(1), 1998, p.\,23--45.

\titidem. \emph{Par"-delà nature et culture}. Paris: Gallimard, 2005.

\titidem. \emph{As Lanças do Crepúsculo: Relações jívaro na
Alta Amazônia}. São Paulo: Cosac \& Naify, 2006 (1994).

\tit{Descola}, Philippe \& \textsc{pálsson}, Gísli. ``Introduction''. In: \textsc{descola}, Philippe \& \textsc{pálsson}, Gísli (orgs.). \emph{Nature and Society: Anthropological Perspectives}. London: Routledge, 1996.

\tit{Dodt}, Gustavo. Descripção dos Rios Parnahyba e Gurupi.
\emph{Brasiliana}, Série 5ª, 138. São Paulo: Companhia Editora Nacional, 1939.

\tit{Dooley}, Robert A. \emph{Vocabulário do Guarani}. Brasília, \textsc{df}: Summer
Institute of Linguistics, 1982.

\tit{Dumont}, Louis. ``The Dravidian Kinship Terminology as Expression of
Marriage''. \emph{Man}, art.\,54, 1953.

\tit{Erikson}, Philippe. ``De l'apprivoisement a l'approvisionnement: chasse,
alliance et familiarisation en Amazonie amérindienne''. In:
\emph{Techniques et cultures}, 9, 1987, p.\,105--140.

\tit{Fausto}, Carlos. \emph{Os Parakanã: dravidianato e casamento avuncular na
Amazônia}. Museu Nacional, dissertação de mestrado, 1991.

\titidem. ``De primos e sobrinhas: terminologia e aliança entre os
Parakanã (Tupi) do Pará''. In: \textsc{viveiros de castro}, Eduardo (org.). \emph{Antropologia do Parentesco -- Estudos Ameríndios}. Rio de Janeiro: Editora~\textsc{ufrj}, 1995.

\titidem. \emph{Inimigos fiéis: história, guerra e xamanismo na
Amazônia}. São Paulo: \textsc{edusp}, 2001.

\titidem. Donos demais: maestria e domínio na Amazônia. \emph{Mana.
Estudos de Antropologia Social}, 14(2), 2008, p\,329--366.

\tit{Feld}, Steven. ``From Ethnomusicology to Echo"-Muse"-Ecology: Reading R.
Murray Schafer in the Papua New Guinea Rainforest''. In: \emph{The Soundscape
Newsletter}, n.\,8, jun. 1994.

\tit{Fernandes}, Florestan. \emph{A Organização Social dos Tupinambá.} São
Paulo: Difel, 1963 (1949).

\tit{Forline}, Louis Carlos. \emph{The persistence and cultural transpormation of the
Guajá indians: foragers of Maranhão State, Brazil}. Doctor
Thesis. University of Florida, 1997.

\tit{Gallois}, Dominique Tilkin. \emph{O movimento na cosmologia Waiãpi: criação, expansão
e tranformação do universo}. Tese de Doutorado. \textsc{ppgas}/\textsc{usp}. São Paulo, 1988.

\titidem. ``Gêneses waiãpi, entre diversos e diferentes''. In:
\emph{Revista de Antropologia}, v.\,50, n.\,1, 1988.

\tit{Forline}, Louis Carlos \& \textsc{garcia} F. Uirá. ``Awá-Guajá: perspectivas para o novo milênio''. In: \textsc{ricardo}, Fany \& \textsc{ricardo}, C. A. (orgs.). \emph{Povos indígenas no Brasil}. São Paulo: \textsc{isa}, 2006.

\tit{Garcia}, Uirá. \emph{Karawara: a caça e o mundo dos Awá-Guajá}. Tese de
Doutorado, Universidade de São Paulo, 2010.

\titidem. ``Ka'á Watá, `andar na floresta': caça e território em
um grupo tupi da Amazônia''. \emph{Mediações -- Revista de Ciências
Sociais}, 17(1), 2012a, p.\,172--190.

\titidem. ``O funeral do caçador: caça e perigo na Amazônia''.
\emph{Anuário Antropológico} (\textsc{ii}), 2012b, p.\,33--55.

\titidem. ``Sobre o poder da criação: parentesco e outras relações
awá-guajá. \emph{Mana. Estudos de Antropologia Social}, v.\,21, 2015, p.\,91--122.

\tit{Gell}, Alfred. \emph{Art and Agency: an Anthropological theory}. Oxford:
Claredon, 1998.

\tit{Goldman}, Marcio. ``Alteridade e experiência: antropologia e teoria
etnográfica''. \emph{Etnográfica -- Revista do Centro de Estudos de
Antropologia Social}, v.\,10, n.\,1, 2006, p.\,161--173.

\tit{Gomes}, Mércio Pereira. ``O povo Guajá e as condições reais para sua
sobrevivência''. In: \emph{Povos Indígenas no Brasil 1987/88/89/90}. São
Paulo: Centro Ecumênico de Documentação e Informação -- \textsc{cedi}, 1991.

\tit{Gonçalves}, Marco Antonio. \emph{O mundo inacabado: ação e criação em uma cosmologia
amazônica}. Rio de Janeiro: Editora \textsc{ufrj}, 2001.

\tit{Gow}, Peter. \emph{Of Mixed Blood: kinship and history in Peruvian
Amazonia}. Oxford: Clarendon, 1991.

\titidem. ``O Parentesco como consciência humana: o caso dos Piro''.
In: \emph{Mana. Estudos de Antropologia Social}, 3(2), 1997, p.\,39--65.

\tit{Grenand}, Pierre. \emph{Ansi Parlaient nos Ancêtres: Essai d'Ethnohistoire
Waiãpi}. Paris: \textsc{orstom}, 1982.

Haraway, Donna. \emph{The Companion Species Manifesto: dogs, people, and
significant otherness.} Chicago: Prickly Paradigm Press, 2003.

Havt, Nadja. \emph{Representações do ambiente e territorialidade entre
os Zo'e/ \textsc{pa}}. Dissertação de Mestrado. São Paulo: \textsc{ppgas}/\textsc{usp}, 2001.

\tit{Hawkes}, Kristen; \textsc{hill}, Kim \& \textsc{o'connell}, James F.
``Why Hunters Gather: Optimal Foraging and the Aché of
Eastern Paraguay''. In: \emph{American Ethnologist}, v.\,9, n.\,2, Economic and Ecological Processes in Society and Culture, maio 1982, p.\,379--398.

\tit{Heckenberger}, Michael, \textsc{neves}, Eduardo G. \& \textsc{Petersen}, James B.
``Como nascem os modelos? As origens e expansões Tupi na Amazônia Central''. In: \emph{Revista de Antropologia}, v.\,41, 1998.

\tit{Hill}, K., \& \textsc{hawkes}, K. ``Neotropical hunting among the Ache of eastern
Paraguay''. In: \textsc{hames}, R. B. \& \textsc{vickers}, W. T. (eds.). \emph{Adaptive
responses of native Amazonians}. Academic Press, 1983, p.\,139--188.

\tit{Holmberg}, Allan R. \emph{Nomads of the long bow -- the Siriono of Eastern
Bolivia}. New York: The American Museum of Natural History -- The
Natural History Press, 1969.

\tit{Hugh"-Jones}, Stephen. ``Bonnes raisons ou mauvaise conscience? De l'ambivalence
de certains Amazoniens envers la consommation de viande''. \emph{Terrain,
26}, 1996, p.\,123--48.

\tit{Huxley}, Francis. \emph{Selvagens amáveis: um antropologista entre os índios
Urubus do Brasil}. Rio de Janeiro: Companhia Editora Nacional, 1963.

\tit{Ingold}, Tim. ``Editorial''. In: \emph{Man -- The Journal of the
Royal Anthropological Institute}, \emph{New Series}, v.\,27, n.\,4, dez. 1992.

\titidem. ``The optimal forager and economic man''. In: \textsc{descola}, Philippe \& \textsc{pálsson}, Gísli (orgs.). \emph{Nature and Society: Anthropological Perspectives}. London: Routledge, 1996.

\titidem. \emph{The perception of the environment: essays in
livelihood, dwelling and skill}. London: Routledge, 2000.

\titidem. ``A Evolução da Sociedade''. In: \textsc{fabian}, A.C. (org.).
\emph{Evolução, Sociedade, Ciência e Universo}. Bauru: Editora da
Universidade do Sagrado Coração, 2003.

\titidem. \emph{Lines: a brief history}. London: Routledge, 2007.

\tit{Ingold}, Tim; \textsc{riches}, David \& \textsc{woodburn}, James (orgs.).
\emph{Hunters and gatherers 1: history, evolution and social change}. Washington D.C: Berg, 1988a.

\titidem. \emph{Hunters and gatherers 2: property, power and
ideology}. Washington D.C: Berg, 1988b.

\tit{Jara}, Fabíola. \emph{El camino del kumu: ecología y ritual entre los
Akuriyó de Surinam.} Quito: Abya"-Yala, 1996.

\tit{Kohn}, Eduardo. ``Animal masters and the ecological embedding of history
among the ávila Runa of Ecuador''. In: \textsc{fausto}, Carlos \& \textsc{heckenberger}, Michael (orgs.). \emph{Time and memory in indigenous Amazonia:
anthropological perspectives}. Gainesville: University Press of Florida, 2007,p.\,106--129.

\titidem. \emph{How forest think. Toward na Anthropology beyond the
Human}. Berkeley/ Los Angeles: University of California Press, 2013.

\tit{Kopenawa}, David \& \textsc{albert} Bruce. \emph{The falling sky: Words of a Yanomami shaman}. Cambridge, \textsc{ma}: Harvard University Press, 2013.

\tit{Kracke}, Waud. \emph{Force and Persuasion: Leadership in na Amazonian
Society}. Chicago: The Universtity of Chicago Press, 1978.

\tit{Laraia}, Roque. \emph{Tupi: Índios do Brasil Atual}. São Paulo: \textsc{fflch},
Universidade de São Paulo, 1986.

\tit{Latour}, Bruno. \emph{Jamais Fomos Modernos: Ensaio de
Antropologia Simétrica.} São Paulo, Editora 34, 1994 (1991).

\tit{Lea}, Vanesa. \emph{Riquezas intangíveis de pessoas partíveis: os
Mebêngôkre (Kayapó) do Brasil Central}. São Paulo: Edusp, 2012.

\tit{Leach}, Edmund. \emph{Edmund Leach} (coletânea de artigos). \textsc{da matta},
Roberto (org.). São Paulo: Ática, 1983.

\tit{Lee}, Richard \& \textsc{richard}, Daly (eds.). \emph{The Cambridge encyclopedia of hunters and gatherers}. Cambridge: Cambridge University Press, 1999.

\tit{Lévi"-Strauss}, Claude. \emph{O pensamento selvagem}. São Paulo:
Companhia Editora Nacional, 1970 (1963).

\titidem. \emph{História de Lince}. São Paulo: Cia. das Letras, 1993.

\titidem. \emph{O cru e o cozido}. São Paulo: Cosac \& Naify, 2004 (1964).

\titidem. \emph{Do mel as cinzas}. São Paulo: Cosac \& Naify, 2004 (1966).

\tit{Lima}, Tânia Stolze. \emph{A parte do Cauim: etnografia Juruna}. Tese de
Doutorado. \textsc{ppgas} -- Museu Nacional/\textsc{ufrj}. Rio de Janeiro, 1995.

\titidem. ``O dois e seu múltiplo: reflexões sobre o perspectivismo
em uma cosmologia tupi''. In: \emph{Mana. Estudos de Antropologia
Social}, 2(2), 1996, p.\,21--47.

\titidem. ``Para uma teoria etnográfica da distinção natureza e
cultura na cosmologia Juruna''. In: \emph{Revista Brasileira de Ciências
Sociais}, v.\,14, n.\,40, jun. 1999.

\titidem. \emph{Um peixe olhou para mim -- O povo Yudjá e a
Perspectiva}. São Paulo: \textsc{isa}/ editora Unesp/ \textsc{n}u\textsc{ti}, 2005.

\tit{Luciani}, José Antonio Kelly. ``Fractalidade e troca de perspectivas''. In: \emph{Mana. Estudos de Antropologia Social}, 7(2), 2001, p.\,95--132.

\tit{Macedo}, Valéria Mendonça. \emph{Nexos da Diferença: Cultura e afecção em uma aldeia
guarani na Serra do Mar}. Tese de Doutorado. São Paulo: Universidade de São Paulo, 2010.

\tit{Magalhães}, Marina Marina Silva. \emph{Aspectos Morfológicos e Morfossintáticos da Língua Guajá}. Dissertação de Mestrado. Universidade de Brasília, 2002.

\titidem. ``Pronomes e Prefixos Pessoais do Guajá''. In: \textsc{rodrigues},
Aryon Dall'Igna \& \textsc{arruda}, Ana Suelly (orgs.). \emph{Novos estudos
sobre línguas indígenas}. Brasília: Editora \textsc{unb}, 2005.

\titidem. \emph{Sobre a Morfologia e a Sintaxe da Língua Guajá
(Família Tupi"-Guarani)}. Tese de Doutorado. Brasília"-\textsc{df}: Universidade de
Brasília, 2007.

\titidem. \emph{Fala de Irakatakôa}. Manuscrito inédito, 2010.

\titidem. Levantamento da documentação existente sobre o povo
indígena Awá-Guajá e registro e sistematização de informações
sociolinguísticas e demográficas atuais. Relatório, \textsc{cgiirc}"-Funai/ \textsc{giz},
Brasília, 2013.

\tit{Maizza}, Fabiana. ``Sobre as crianças"-planta: o cuidar e o seduzir no
parentesco Jarawara''. In: \emph{Mana. Estudos de Antropologia Social}, 20(3), 2014, p.\,491--518.

\tit{Martins}, Marlúcia Bonifácio \& \textsc{Oliveira}, Tadeu Gomes de (orgs.).
\emph{Amazônia Maranhense: diversidade e conservação}. Belém: Museu Paraense Emílio Goeldi, 2011.

\tit{Müller}, Regina Pólo. \emph{Os Asuriní do Xingu: história e arte}. Campinas:
Ed.Unicamp, 1990.

\tit{Nimuendajú}, Curt. ``The Guajá, by Curt Nimuendajú''. In: \textsc{steward}, Julian
Haynes (org.). \emph{Handbook of South American Indians v. 3}.
Washington: Govt. Print. Off, 1948.

\titidem. \emph{As lendas da criação e destruição do mundo
como fundamentos da religião dos Apapocúva"-Guarani}. São Paulo:
Hucitec/Edusp, 1987 (1914).

\tit{Neves}, Eduardo Góes. ``O Velho e o Novo na Arqueologia Amazônica''. \emph{Revista
\textsc{usp}}, Brasil, v.\,44, 1999, p.\,87--113.

\tit{Noelli}, F.S. ``As hipóteses sobre o centro de origem e rotas de
expansão dos Tupi'', \emph{Revista de Antropologia}, 39(2), 1996, p.\,7--53.

\tit{O'dwyer}, Eliane Cantarino. \emph{Laudo antropológico -- área indígena Awá}. Fundação
Nacional do Índio (Funai), 2001.

\titidem. \emph{O papel social do Antropólogo}. Rio de Janeiro: Laced/ e"-papers, 2010.

\tit{Overing}, Joanna. ``O Fétido odor da morte e os aromas da vida. Poética dos
saberes e processo sensorial entre os Piaroa da Bacia do Orinoco''. In:
\emph{Revista de Antropologia}, v.\,49, n.\,1, 2006.

\tit{Overing"-Kaplan}, Joanna. \emph{The Piaora -- a People of the Orinoco Basin: A Study
in Kinship and Marriage}. Oxford: Claredon Press, 1975.

\tit{Pissolato}, Elisabeth. \emph{A Duração da Pessoa: mobilidade, parentesco e
xamanismo mbya (guarani)}. São Paulo: Editora Unesp, 2007.

\tit{Ramos}, Alcida R. ``Ethnology Brazilian Style''. \emph{Cultural Anthropology}, v.\,5, n.\,4, 1990, p.\,452--472.

\tit{Ribeiro}, Darcy. \emph{Uirá sai à procura de Deus: Ensaios de Etnologia e
Indigenismo}. Rio de Janeiro: Paz \& Terra, 1980.

\titidem. \emph{Diários Índios -- Os Urubu"-Kaapor}. São Paulo:
Companhia das Letras, 1996.

\tit{Rival}, Laura M. ``The Growth of Family Tress: Huaorani Conceptualization
of Nature and Society''. \emph{Man}, 28(4), 1993, p.\,635--52.

\titidem. ``Androgynous parents and guest children: the Huaorani
couvade''. \emph{Journal of the Royal Anthropological Institute}, 4(4), 1998, p.\,619--42.

\titidem. ``Introduction: South America''. In: \textsc{lee}, Richard \& \textsc{daly}, Richard. \emph{The Cambridge encyclopedia of hunters and gatherers}.
Cambridge: Cambridge University Press, 1999.

\titidem. \emph{Trekking Through History -- the Huaorani of
Amazonian Ecuador}. New York: Columbia University Press, 2002.

\tit{Rivière}, Peter. \emph{Marriage Among The Trio: A Principle of Social
Organization}. Oxford: Clarendon Press, 1969.

\tit{Rodrigues}, Aryon Dall'Igna. ``Relações internas na família linguística Tupi
Guarani''. São Paulo: \emph{Revista de Antropologia}, 1984-85, p.\,27--28.

\tit{Roosevelt}, Anna C.R. ``Ancient and Modern Hunter"-Gatherers of Lowland South
America: An Evolutionary Problem''. In: \textsc{balée}, William (org.).
\emph{Principles of Historical Ecology}. New York: Columbia
University Press, 1992, p.\,190--212.

\tit{Rosalen}, Juliana. \emph{Aproximação à temática das \textsc{dst} junto aos Wajãpi do
Amapari. Um estudo sobre malefícios, fluidos corporais e sexualidade}.
Dissertação de Mestrado. Universidade de São Paulo, 2005.

\tit{Sahlins}, Marshall. ``O `pessimismo sentimental' e a experiência
etnográfica: por que a cultura não é um `objeto' em via de extinção
(parte \textsc{i})''. \emph{Mana. Estudos de Antropologia Social}, v.\,3,
n.\,1, abr. 1997a.

\titidem. ``O `pessimismo sentimental' e a experiência
etnográfica: por que a cultura não é um `objeto' em via de extinção
(parte \textsc{ii}). \emph{Mana. Estudos de Antropologia Social}, v.\,3,
n.\,2, p.\,103--150, out. 1997b.

\titidem. ``What kinship is (part one)''. \textsc{jrai} (\textsc{ns}), 17, p.\,2--19.
``What kinship is (part two)'', \textsc{jrai} (\textsc{ns}), 17, p.\,227--42, 2011.

\tit{Santos}, Rosana de Jesus Diniz. \emph{Awa Papejapoha: um estudo sobre educação escolar
entre os awá guajá/\textsc{ma}}. Monografia de especialização em Desenvolvimento
e relações sociais no campo: diversidade e interculturalidade dos povos
originários, comunidades tradicionais e camponesas do Brasil,
Universidade de Brasília, 2015.

\tit{Schneider}, David M. \emph{American kinship: a cultural account}. Chicago: The
University of Chicago Press, 1968.

\tit{Schroeder}, Ivo. \emph{Política e parentesco nos} Xerente. Tese de
doutorado, Universidade de São Paulo, 2006.

\tit{Seeger}, Anthony. \emph{Nature and Society in Central Brazil: The Suyá
Indians of Mato Grosso}. Cambridge, Mass.: Harvard University Press, 1981.

\tit{Seeger}, Anthony; \textsc{da matta}, Roberto A. \& \textsc{viveiros de Castro}, Eduardo. ``A construção da pessoa nas sociedades indígenas
brasileiras''. \emph{Boletim do Museu Nacional}, 32, 1979, p.\,2--19.

\tit{Service}, Elman R. \emph{Os caçadores}. Rio de Janeiro: Zahar Editores,
1971 (1966).

\tit{Sigrist}, Tomas. \emph{Guia de Campo: aves da Amazônia Brasileira}. São
Paulo: Avisbrasilis, 2008.

\tit{Silva}, Márcio Ferreira. ``Sistemas Dravidianos Amazônicos: o caso
waimiri"-atroari''. In: \textsc{viveiros de castro}, Eduardo (org.).
\emph{Antropologia do Parentesco -- Estudos Ameríndios.} Rio de Janeiro:
Editora \textsc{ufrj}, 1995a.

\titidem. \emph{Sistemas de aliança na América do Sul Tropical:
modelos e práticas}. Manuscrito inédito, 2005.

\tit{Silverwood"-Cope}, Peter L. \emph{Os Makú: povo caçador do noroeste da Amazônia}.
Brasília: Editora UnB, 1990.

\tit{Sponsel}, Leslie E. ``The Human Niche in Amazonia : Explorations in
Ethnoprimatology''. In: \textsc{kinzey}, Warren G. (org.). \emph{New World Primates:
Ecology, Evolution, and Behavior}. New York, \textsc{ny}: Aldine de Gruyter, 1997, p.\,143-165.

\tit{Stearman}, Allyn MacLean. \emph{Yuquí: Forest Nomads ina a Changing World.}
San Francisco: Holt, Rinehart \& Winston, 2001 (1989).

\titidem. ``Neotropical Indigenous Hunters and tehir neighbors.
Sirionó, Chimane, and yuquí Hunting on the Bolivian Frontier''. In:
\textsc{redford}, Kent H \& \textsc{padoch}, Christine (orgs.). \emph{Conservation of
neotropical forests: working from traditional resource use}. New York:
Columbia University Press, 1992.

\tit{Strathern}, Marilyn. ``No Nature, no culture: the Hagen case''. In: \textsc{maccormack}, C. \& \textsc{strathern}, M. (eds.). \emph{Nature, Culture, and Gender}.
Cambridge: University of Cambridge Press, 1980, p.\,174--222.

\titidem. \emph{The Gender of the Gift: Problems with Women and
Problems with Society in Melanesia}. Berkeley: University of California
Press, 1988.

\titidem. \emph{The relation: issues in complexity and scale}.
Cambridge: Prickly Pear Press, 1995a.

\titidem. ``Necessidade de Pais, Necessidade de Mães''. In: \emph{Revista
de Estudos Feministas}, v.\,3, n.\,2, 1995b.

\titidem. \emph{Property, substance and effect: anthropological
essays on persons and things}. London: The Athlone Press, 1999a.

\titidem. ``No limite de uma certa linguagem (entrevista)''. In:
\emph{Mana. Estudos de Antropologia Social}, 5(2), p.\,157--175, 1999b.

\titidem. \emph{O Gênero da Dádiva -- Problemas com as
mulheres e problemas com a sociedade na Melanésia}. Campinas: Editora da
Unicamp, 2006 (1988).

\titidem. ``From Papua New Guinea to a \textsc{uk} Council on Bioethics:
fieldwork at the beginning and end of an anthropological lifetime''.
Mimeo, Opening lecture for \textsc{xii} \textsc{gec} (`students in the field') conference,
São Paulo, \textsc{usp}, 2014a.

\titidem. Reading relations backwards. \emph{The Journal of the
Royal Anthropological Institute} (n.s.), v.\,20, p\,3--19, 2014b.

\tit{Sztutman}, Renato. \emph{O profeta e o principal}. São Paulo: Edusp, 2012.

\tit{Taylor}, Anne"-Christine. ``Remembering to forget: identity, mourning and memory
among the Jívaro''. \emph{Man}, 28(4), p.\,653--78, 1993.

\titidem. ``The Soul's Body and it's States: An Amazonian
Perspective on the Nature of Being Human''. In: \emph{The Journal of the Royal Anthropological Institute}, v.\,2, n.\,2, p.\,201--215, 1996.

\titidem. ``Wives, Pets and Affines: Mariage among Jivaro''. In:
\textsc{rival}, Laura \& \textsc{whitehead}, Neil L. (orgs.). \emph{Beyond the Visible and
the Material -- the amerindization of society in the work of Peter
Rivière.} Oxford: Oxford University Press, 2001.

\tit{toral}, André. ``Caminhando só: Comentários sobre o filme Serra da
desordem (2006), de Andréa Tonacci''. In: \emph{Facom}, n.\,17, 2007.

\tit{Trautmann}, T. R \& \textsc{barnes}, R. H. ``Dravidiam, Iroquis ans Crow"-Omaha in North American Perspective'' In: \textsc{godelier}, M. Trautmann \& \textsc{tjon sie fat}, F. E. \emph{Transformations of Kinship}. Washington: Smithsonian Institution Press, 1998.

\tit{vander velden}, Felipe Ferreira. \emph{Inquietas Companhias: sobre os animais de criação entre os Karitiana}. Tese de Doutorado. Campinas: Universidade de
Campinas, 2010.

\tit{Vaz}, Antenor. ``Isolados no Brasil -- Políticas de Estado: Da tutela
para a Política de Direitos -- Uma Questão Resolvida?''. In: Informe 10,
\textsc{iwgia}. Brasília: Grupo Internacional de Trabalho sobre Assuntos
Indígenas, 2011.

\titidem. ``Povos Indígenas Isolados e de Recente Contato no Brasil
-- A que será que se destinam?''. In: \emph{Le monde Diplomatique Brasil}, ago. 2014.

\tit{viveiros de castro}, Eduardo. ``Bibliografia etnológica básica tupi"-guarani''. In:
\emph{Revista de Antropologia}, v.\,27/28. São Paulo: \textsc{usp}, 1984/85.

\titidem. \emph{Araweté: os deuses canibais}. Rio de Janeiro: Ed. Jorge
Zahar, 1986.

\titidem. \emph{From the Enemies Point of View: Humanity and
Divinity in an Amazonian Society}. Chicago: The University of Chicago
Press, 1992.

\titidem. ``Alguns aspectos da afinidade no dravidianato amazônico''.
In: \textsc{viveiros de castro}, E. \& \textsc{cunha}, M. Carneiro da (orgs.). \emph{Amazônia: etnologia e história indígena}. são Paulo: \textsc{nhii}"-\textsc{usp}/ \textsc{fapesp}, 1993, p.\,150--2010.

\titidem. ``Os pronomes cosmológicos e o perspectivismo ameríndio.
In: \emph{Mana. Estudos de Antropologia Social}, 2(2), 1996, p.\,115--144.

\titidem. \emph{A inconstância da alma selvagem -- e outros ensaios
de antropologia}. São Paulo: Cosac \& Naify, 2002.

\titidem. ``Filiação Intensiva e Aliança Demoníaca''. In:
\emph{Revista Novos Estudos, Cebrap}, 2007.

\titidem. ``Xamanismo transversal: Lévi"-Strauss e a cosmopolítica
amazônica''. In: \textsc{caixeta de queiroz}, Ruben \& \textsc{nobre}, Renarde Freire
(orgs.). \emph{Lévi"-Strauss: leituras brasileiras}. Belo Horizonte:
Editora \textsc{ufmg}, 2008.

\titidem. ``The Gift and the Given: three nano"-essays on kinship
and magic''. In: \textsc{bamford}, S. \& \textsc{leach}, J. (orgs.). \emph{Kinship and Beyond: the genealogical model reconsidered}. Oxford: Bergham Books, 2009.

\tit{Wagley}, Charles. \emph{Lágrimas de Boas Vindas: os índios Tapirapé
do Brasil Central}. São Paulo: Itatiaia/\textsc{usp}, 1988 (1977).

\tit{Wagley}, Charles \& \textsc{galvão}, Eduardo. \emph{Os índios Tenetehara: uma cultura em transição}. Rio de Janeiro: \textsc{mec}, 1961.

\tit{Wagner}, Roy. \emph{A invenção da cultura}. São Paulo, Cosac \& Naify, 2010 (1981).

\tit{Willerslev}, Rane. \emph{Soul Hunters: hunting, animism, and personhood
among the siberian yukaghirs}. Berkeley, Los Angeles, London: University
of California Press, 2007.

\tit{Yokoi}, Marcelo. \emph{Na Terra, no céu: os Awá-Guajá e os Outros}.
Dissertação de Mestrado. São Carlos: Universidade Federal de São Carlos, 2014.

\section{Trabalhos acadêmicos referentes os Awá-Guajá consultados}

\tit{Balée}, William. \emph{Culturas da floresta: apontamentos críticos
sobre a ecologia cultural na Amazônia}, trabalho lido no Simpósio
\textsc{aba}/\textsc{anpocs}, 1987.

\titidem. ``O povo da capoeira velha: caçadores"-coletores
das terras baixas da América do Sul''. Trabalho apresentado na
Conferência Amazônica da Fundação Memorial da América Latina, 25 mar. 1992.

\titidem. ``People of the Fallow: A historical Ecology of
Foraging in Lowland South América''. In: \textsc{redford}, Kent H \& \textsc{padoch}, Christine (orgs.). \emph{Conservation of neotropical forests: working from
traditional resource use}. New York: Columbia University Press, 1992.

\titidem. \emph{Footprints of the forest: Ka'apor
ethnobotany -- the historical ecology of plant utilization by an
amazonian people}. New York: Columbia University Press, 1994.

\titidem. ``Language, law, and land in Pre"-Amazonian
Brazil''. In: \emph{Texas International Law Journal}, v.\,32, n.\,1, 1997.

\titidem. ``Antiquity of traditional ethnobiological
knowledge in Amazonia: the Tupí-Guaraní family and time''. In:
\emph{American Society for Ethnohistory}. Tulane University, 2000.

\tit{Beghin}, François"-Xavier. \emph{Les Guajá}. Letter to Darcy
Ribeiro, 15 jul. 1950.

\titidem. ``Les Guajá''. \emph{Revista do Museu Paulista} (n.s), 5, p.\,137--39, 1951.

\titidem. ``Relation du premier contact avec les indiens
Guajá''. \emph{Journal de la Société des Américanistes} (n.s.), \textsc{xlvi}, p.\,197--204, 1957.

\tit{Cardoso}, Guilherme Ramos. \emph{Uma leitura sobre identidade e
etnicidade na literatura sobre os Awá-Guajá}. Dissertação de mestrado.
Universidade Federal Fluminense, 2013.

\tit{Cormier}, Loretta. \emph{Kinship With Monkeys}. New York: Columbia
University Press, 2003.

\titidem. ``Um aroma no ar: a ecologia histórica das
plantas anti"-fantasma entre os Guajá da Amazônia''. In: \emph{Mana.
Estudos de Antropologia Social}, 11(1), p.\,129--154, 2005.

\tit{Cunha}, Péricles. \emph{Análise fonêmica preliminar da língua
Guajá}. Dissertação de Mestrado. Universidade de Campinas, 1987.

\tit{Forline}, Louis Carlos. The persistence and cultural transformation
of the Guaja Indians: Foragers of Maranhao state, Brazil. Tese de
doutorado. University of Florida, Gainesville, 1997.

\titidem. \emph{Por uma Síntese Biocultural:
relatório preliminar sobre as comunidades Guajá dos Postos Indígenas
awá, Tiracambu, Juriti e Guajá.} Reno: Universidade de Nevada, 2007.

\titidem. ``The body social and the body private: fine
tuning our understanding of partible paternity and reproductive
strategies among amazonian indigenous groups''. In: \emph{Proceedings of
the Southwestern Anthropological Association}, v.\,5, p.\,1--8, 2011.

\tit{Hernando}, Almudena \emph{et al}. ``Gender, Power, and Mobility among the Awá-Guajá
(Maranhão, Brazil)''. In: \emph{Journal of Anthropological Research}, v.\,67, 2011.

\tit{Hernando}, Almudena \& \textsc{beserra coelho}, Elisabeth (org.).
\emph{Estudos sobre os Awá -- caçadores"-coletores em transição.} São
Luís: Edufma, 2013.

\tit{Magalhães}, Marina Silva. \emph{Aspectos Morfológicos e
Morfossintáticos da Língua Guajá}. Dissertação de Mestrado. Universidade
de Brasília, 2002.

\titidem. ``Pronomes e Prefixos Pessoais do Guajá''. In: \textsc{rodrigues}, Aryon Dall'Igna \& \textsc{arruda}, Ana Suelly (orgs.).
\emph{Novos estudos sobre línguas indígenas}. Brasília: Editora \textsc{unb}, 2005.

\titidem. \emph{Sobre a Morfologia e a Sintaxe da
Língua Guajá (Família Tupi"-Guarani)}. Tese de Doutorado. Universidade de
Brasília, 2007.

\titidem. \emph{Fala de Irakatakôa}. Manuscrito inédito, 2010.

\tit{Nimuendajú}, Curt. The Guajá. In: \emph{Handbook of South American
Indians}, v.\,3. Washington, D.C.: U. S. Government Printing Office, p.\,135--36, 1949.

\tit{oliveira}, Silviene Fabiana de \emph{et al}.  ``The Awá-Guajá Indians of
the Brazilian Amazon. Demographic Data, sérum Protein Markers and Blood
Groups''. In: \emph{Hum Hered}, 48, 1998, p.\,163--168.

\tit{O'Dweyer}, Eliane Cantarino. \emph{Laudo antropológico -- área
indígena Awá}. Fundação Nacional do Índio (Funai), 2001.

\titidem. \emph{O papel social do Antropólogo}.
Rio de Janeiro: Laced/e"-papers, 2010.

\tit{Prado}, Helbert Medeiros. \emph{O impacto da caça versus a conservação de
primatas numa comunidade indígena Guajá}. Dissertação de Mestrado.
Universidade de São Paulo, 2007.

\tit{Prado}, Helbert Medeiros; \textsc{forline}, Louis Carlos \& \textsc{kipnis}, Renato.
``Hunting Practices among the Awá-Guajá: Towards a long"-term analysis of
sustainability in an Amazonian indigenous community''. In: \emph{Bol. Mus.
Para. Emilio Goeldi. Cienc.Hum.}, Belém, v.\,7, n.\,2, p\,479--491, maio"-ago. 2012.

\tit{Rubial}, Alfredo González; \textsc{hernando}, Almudena \& \textsc{politis}, Gustavo.
``Ontology of the self and material culture: Arrow"-making among the Awá
hunter"-gatherers (Brazil)''. In: \emph{Journal of Anthropological
Archeology}, 2010.

\tit{Rubial}, Alfredo González \emph{et al.} ``Domestic Space and Cultural Transformation
Among the Awá of Eastern Amazonia''. In: \emph{\textsc{bar} International Series
2183}/ \emph{Archaeological Invisibility and Forgotten Knowledge:
Conference Proceedings}, Łódź, Poland, 5--7 set. 2007.
\end{bibliohedra}


\section{Documentos oficiais consultados}

\begin{itemize}
{\footnotesize
\item[1961] Decreto nº 51.026 de 25/7/61, Cria a Reserva Florestal do
Gurupi e dá outras providências. Jânio Quadros (Presidente do Brasil).

\item[1973] Funai -- \emph{Relatório da viagem ao alto rio Carú e
igarapé da Fome para verificar a presença de índios Guajá}. Valéria
Parise. São Luís.

\item[1973] Funai -- \emph{Relatório de viagem -- frente de atração
Guajá 6ª\textsc{dr} do Maranhão}. João Fernandes Moreira.

\item[1973] Funai -- \emph{Proposta para Interdição da Área Awá/Guajá
4ª \textsc{suer} Maranhão.} Fiorello Parise.

\item[1980] \emph{Parecer sobre o processo Funai/ \textsc{bsb}/ 5044/ 79,
referente aos índios Guajá do Estado do Maranhão}. Mércio Pereira Gomes.

\item[1980] \emph{Relatório sobre o contato e a necessidade de
transferência de 27 índios Guajá do Igarapé Timbira, município de santa
Luzia, para a Reserva Caru, Município de Bom Jardim, estado do Maranhão.
E da necessidade de se criar um novo posto para esses e outros Guajá que
se encontram na Reserva Caru, e da Implementação de uma Política
indigenista própria para esses índios para que se possa evitar seu
declínio populacional}. Mércio Pereira Gomes.

\item[1980] Centro Ecumênico de Documentação e Informação (\textsc{cedi}) --
\emph{Ficha sobre a situação atual das populações indígenas no Brasil.}
Carlos Ubbiali.

\item[1980] \emph{Carta de Mércio Pereira Gomes ao \textsc{cedi} (Beto Ricardo
e Fany Ricardo}).

\item[1981] Funai -- \emph{Relatório Preliminar sobre a Situação de
Grupos Guajá que se encontram fora de Reservas Indígenas e que precisam
de uma solução em caráter de urgência}. Mércio Pereira Gomes. São Luís.

\item[1982] Funai -- \emph{A problemática Indígena no Maranhão,
especificamente nas áreas de influência imediata da ferrovia Carajás:
reserva Turiaçu, reserva Caru e reserva Pindaré}. Mércio Pereira Gomes.

\item[1982] Comissão Pró-Índio do Maranhão -- \emph{Nota de Repúdio.}
Elizabeth Maria Beserra de Coelho.

\item[1982] \emph{Resposta ao Del. da 6ª \textsc{dr}"-Funai com acusações em
relação ao tratamento que a Funai dispensa aos Guajáe outros povos --
\textsc{ma}}. Mércio Pereira Gomes.

\item[1983] Funai -- \emph{Relatório de viagem da frente de atração
Guajá no período de 03/06/82 e 12/06/82}. Fiorello Parise. Funai.

\item[1983] Funai -- \emph{Relatório anual 1983 -- Postos de
Vigilância/ Frente de Atração Guajá}. Cornélio Vieira de Oliveira.

\item[1984] Funai -- \emph{Relatório sobre debelação de invasão}.
Chefe de posto Guajá e Alto Turiaçú.

\item[1984] Funai -- \emph{Relatório sobre a Frente de Atração}. José
Araújo Filho.

\item[1984] \emph{Carta ao Exmo. Senhor Dr.Nelson Marabuto,
Presidente da Funai}/ \emph{Programa Awá}. José Porfírio Fontenele de
Carvalho.

\item[1985] \emph{Certificado do cartório de Carutapera}. Januário de
Sena Loureiro.

\item[1985] Funai -- \emph{Memorial Descritivo de delimitaçãoo --
Área Indígena Awá.} Autor desconhecido.

\item[1985] Funai/\textsc{cvrd} -- \emph{Relatório sobre os índios Guajá
Próximos à ferrovia Carajás -- Km 400}. Mércio Pereira Gomes.

\item[1985] Funai -- \emph{Relatório do reconhecimento da área da
Serra da desordem}. José Carlos dos Reis Meirelles Júnior.

\item[1985] Funai -- \emph{Relatório de viagem}. Sérgio de Campos
(Eng. Agrimensor).

\item[1985] Funai/\textsc{cvrd} -- \emph{Programa Awá -- relatório Inicial}.
Mércio Pereira Gomes.

\item[1985] Funai/\textsc{cvrd} -- \emph{Relatório antropológico sobre a área
indígena Guajá (Awá-Gurupi)}. Mércio Pereira Gomes.

\item[1985] Funai/ \textsc{cvrd} -- \emph{Área indígena Awá-Gurupi. Estudos e
proposta.} José Carlos Meirelles Júnior e Mércio Pereira Gomes.

\item[1985] \textsc{cvrd} -- \emph{Nota para a imprensa.}

\item[1987] Funai -- \emph{Parecer nº 171/87 -- \textsc{gt}.Interministerial
-- \textsc{dec}. nº 94.945/87. Área Indígena Awá.} Romero Jucá Filho \emph{et
al}.

\item[1987] Funai -- \emph{Descrição do Perímetro. Anexo Área Indígena
Awá/Guajá}. Fiorello Parise.

\item[1987] \textsc{cimi} -- \emph{Parecer sobre os Awá} . Paulo Mahado
Guimarães.

\item[1987] \textsc{suer}/ \textsc{spag} -- Convênio Funai/\textsc{cvrd}. \emph{Relatório de
atividades de agosto, setembro e outubro de 1987}. Fiorello Parise.

\item[1987] Funai -- \emph{Carta referente a criação do Sistema de
Proteção}. Fiorello Parise.

\item[1987] Funai -- \emph{Relatório de atividades junho/ julho de
1987}. Fiorello Parise.

\item[1987] Funai -- \emph{Relatório de atividades de junho a dezembro
de 1987}. Domingos Faria Pereira.

\item[1987] Funai -- \emph{Área indígena Awá (Declaração de Ocupação
Indígena)}, ref: \textsc{proc}. Funai/\textsc{bsb}/2582/85. Itagiba Christiano de O.C.
Filho.

\item[1987] Funai -- \emph{Portaria nº 3.767, de 13 de novembro de 1987.} Romero Jucá Filho.

\item[1987] Funai -- \emph{Memorial descritivo de delimitação, Área
Indígena Awá}. Cornélio Oliveira, Artur N. Mendes e Reinaldo Florindo.

\item[1987] \textsc{cedi} -- \emph{Campanha Guajá}.

\item[1988] Funai -- \emph{Parecer nº 197/88 -- \textsc{gt} Interministerial
-- \textsc{dec}.nº. 94.945/87}. Romero Jucá Filho \emph{et al}.

\item[1988] \emph{Portaria Interministerial nº 076} de 03 de maio de
1988, de posse permanente da área Awá. João Alves Filho; Jader
Fontenelle Barbalho.

\item[1988] \emph{Portaria Interministerial nº 158, de 08 de setembro
de 1988}.

\item[1988] Funai -- \emph{Carta de Santa Inês}. Sidney Possuelo, et
al.

\item[1988] Funai -- \emph{Relatório de atividades do programa
Awá-Guajá, convênio Funai/ \textsc{cvrd}, novembro, dezembro/ 87 e considerações
finais}. Fiorello Parise.

\item[1988] Funai -- \emph{Relatório de atividades meses janeiro e
fevereiro/ 88}. Fiorello Parise.

\item[1988] Funai -- \emph{Relatório de atividades do Convênio
Funai/\textsc{cvrd}, referente ao 1ºtrimestre de 1988}. Idalércio de Andrade
Moreira.

\item[1988] Funai -- \emph{Relatório de viagem às \textsc{adr}'s do Estado do
Maranhão}. Leda Aparecida Câmara de Azevedo \& Marta Luciana de Sá Pinto
Barbosa.

\item[1988] Funai -- \emph{Relatório de atividades meses janeiro e
fevereiro/ 88}. Fiorello Parise.

\item[1988] Funai -- \emph{Assessoria de índios isolados da 4ª Suer
-- junho 88}. Fiorello Parise.

\item[1988] Funai -- \emph{Relatório de Atividades da Assessoria de
Índios Isolados}. Dinarte Nobre de Madeiro.

\item[1988] Funai -- \emph{Relatório de atividades da Assessoria de
Índios Isolados da 4ª \textsc{suer} referente ao mês de junho}. Fiorello Parise.

\item[1988] Funai -- \emph{Relatório de Atividades do Sistema de
Proteção Awá-Guajá}. Dimas Valente.

\item[1988] Funai -- \emph{Relatório final de atividades de 1988, e
panorama sobre a situação da área de índios isolados da 4ª \textsc{suer}}.
Fiorello Parise.

\item[1988] Funai -- \emph{Relatório} \emph{de atividades do mês de
outubro/88 do sistema de proteção Awá/Guajá}. Dimas Valente.

\item[1988] Funai -- \emph{Comunicação -- índios Guajá sem contato na
sede do \textsc{pinc}}. Sabino Francisco Conceição Neto.

\item[1989] Funai -- \emph{Relatório sobre aspectos gerais do \textsc{pin}
Guajá e algumas sugestões de trabalho}. Egipson Nunes Correia.

\item[1989] Funai -- \emph{Parecer sobre o relatório sobre aspectos
gerais do \textsc{pin} Guajá e algumas sugestões de trabalho}. Samuel Vieira
Cruz.

\item[1989] Funai -- \emph{Relatório de levantamento da área indígena
Caru e Alto Turiaçu.} Fiorello Parise \emph{et al}.

\item[1989] Funai -- \emph{Situação do Sistema de Proteção
Awá/Guajá}. Samuel Vieira Cruz.

\item[1989] Funai -- \emph{Viagem à Terra Sem Lei -- relatório de
levantamento da \textsc{ai} Awá: equipes nº 02 e 03}. Fiorello Parise.

\item[1989] Funai -- \emph{Proposta de trabalho para a vigilância da
área awá}. Samuel Vieira Cruz.

\item[1990] Funai -- Viagem ao Sistema de Proteção Awa/Guajá.
Fiorello Parise.

\item[1990] Funai -- \emph{Apresentação de relatório, \textsc{pin} Guajá}.
Egipson Nunes Correia.

\item[1990] Convênio \textsc{cvrd} -- Funai. Missão do Banco Mundial -- \textsc{papp}/
\textsc{ma}. Cornélio Vieira de Oliveira.

\item[1990] Funai -- \emph{Informações sobre o grupo Miri"-Miri, da
área Awá-Guajá}. Samuel Vieira Cruz.

\item[1990] Funai -- \emph{Relatório de atividades do sistema de
proteção Awá/Guajá no exercício de 1990.} Renildo Matos dos Santos.

\item[1999] \emph{Relatório Final -- Assessoria Antropológica junto aos
Awá-Guajá}. Funai/ \textsc{cvrd}. Nadja Havt Bindá.

\item[2000] \emph{Assessoria Antropológica -- Relatório Final}. Funai/
\textsc{cvrd}. Nadja Havt Bindá.

\item[2001] \emph{Laudo antropológico}. Processo judicial no
95.353--8. 5a Vara da Justiça Federal do Maranhão. Eliane Cantarino
O'Dwyer.

\item[2001] \emph{Diagnóstico sócio"-econômico e geoambiental das
Terras Indígenas do Pindaré, caru, Awá e Alto Turiaçu}. Elizabeth Maria
Beserra Coelho, Rogério Tavares Pinto e Kátia Núbia Ferreira Correa.

\item[2002] Funai -- \emph{Relatório Awá-Guajá 2002}. Mércio Pererira
Gomes \& José Carlos Meireles. Petrópolis.

\item[2002] Universidade federal de são Paulo/ Escola Paulista de
Medicina. \emph{Descrição das precariedades atuais relativas à saúde e
necessidades dos índios Guajá}. João Paulo Botelho Vieira Filho.

\item[2006] Funai -- \emph{Relatório de Atividades Awá-Guajá, maio de
2006}. Maurício de Lima Wilke.

\item[2009] Expedição da Funai confirma a existência de índios
isolados no Maranhão. Site da Funai.

\item[2013] Funai -- Relatório de Incursão à Campo -- Programa Awá
Guajá. Nelson Cesar Destro Junior.

\item[2013] Funai -- Relatório sobre viagem para a \textsc{ti} Awá. Felipe V.M.
Almeida.

\item[s/d] Funai -- \emph{Relatório sobre a Frente de Atração
Awá-Guajá}. Domingos Faria Pereira.

\item[s/d] Funai -- \emph{Carta a Sidney Possuelo}. Domingos Faria
Pereira.

\section{Filmografia}

\item[2006] Tonacci, Andrea. \emph{Serras da Desordem}. Longa Metragem.
}
\end{itemize}

