%!TEX  root=./LIVRO.tex

\chapter{Referências
Bibliográficas}\label{referuxeancias-bibliogruxe1ficas}

Albert, Bruce.

\textbf{1995}. "O ouro canibal e a queda do céu: uma crítica xamânica da
economia política da natureza". In: \emph{Série Antropologia},
\emph{174}. Brasília.

Albert, Bruce \& Miliken, William.

\textbf{2009}. \emph{Urihi A: a terra-floresta Yanomami}. São Paulo:
ISA-IRD.

Almeida, Mauro W. B.

\textbf{2007}. \emph{Caipora e outros conflitos ontológicos} (mimeo).
Conferência Quartas Indomáveis, São Carlos. Disponível em:
http://mwba.files.word-
press.com/2010/06/2008-almeida-caipora-e-outros-conflitos-ontologicos-\_sao-carlos-
-rev-\_2010-02.pdf. Acesso em 06/2012.

AmaZone, Projeto.

\textbf{2010}. \emph{A onça e a diferença}. Disponível em
\textless{}http://amazone.wikia.com/wiki/Projeto\_AmaZone\textgreater{}.
Acesso em: 24 nov. 2010.

Andrade, Lucia M. M.

\textbf{1992}. \emph{O Corpo e os Cosmos: Relações de Gênero e o
Sobrenatural entre os Asurini do Tocantins}. Dissertação de Mestrado.
São Paulo: Universidade de São Paulo.

Århem, Kaj.

\textbf{2001}. \emph{La red cósmica de la alimentación}. In Descola,
Philppe \& Gislí Palson. México: Siglo XXI.

Asad, Talal.

\textbf{1986}. "The Concept of Cultural Translation in British Social
Anthropology". In: James Clifford and George E. Marcus (eds.).
\emph{Writing Culture. The Poetics and Politics of Ethnography}.
Berkley: University of California Press.

Balée, William.

\textbf{1987}. \emph{Culturas da floresta: apontamentos críticos sobre a
ecologia cultural na Amazônia}, trabalho lido no Simpósio ABA/ANPOCS.

\textbf{1992}. ``O povo da capoeira velha: caçadores-coletores das
terras baixas da América do Sul''. Trabalho apresentado na Conferência
Amazônica da Fundação Memorial da América Latina em 25 de Março.

\textbf{1992}. ``People of the Fallow: A historical Ecology of Foraging
in Lowland South América''. In Redford, Kent H \& Christine Padoch
(orgs). \emph{Conservation of neotropical forests: working from
traditional resource use}. New York: Columbia University Press.

\textbf{1994}. \emph{Footprints of the forest: Ka'apor ethnobotany --
the historical ecology of plant utilization by an amazonian people}. New
York: Columbia University Press.

\textbf{1997}. ``Language, law, and land in Pre-Amazonian Brazil''. In
\emph{Texas International Law Journal} vol. 32 (nº1, winter).

1999. ``The Sirionó of the Lanos de Mojos, Bolívia'' in: Daly, Richard
\& Richard Lee (orgs). \emph{The Cambridge Encyclopedia of Hunters and
Gatherers}. Cambridge: Cambridge University Press.

\textbf{2000}. ``Antiquity of traditional ethnobiological knowledge in
Amazonia: the Tupí-Guaraní family and time''. In \emph{American Society
for Ethnohistory} - Tulane University (Spring).

\textbf{2013}. \emph{Cultural Forests of the Amazon}. \emph{A Historical
Ecology of People and their Landscapes}. Tuscaloosa, Alabama: The
University of Alabama Press.


Barbosa, Gabriel Coutinho.

\textbf{2007}. \emph{Os Aparai e Wayana e suas redes de intercâmbio}.
Tese de Doutorado. São Paulo: Universidade de São Paulo.

Bateson, Gregory.

\textbf{1956 (2000)}. ``Toward a Theory of Schizophrenia''. In
\emph{Steps to na ecology of mind.} Chicago/London: The University of
Chicago Press.

Beckerman, S. and P. Valentine, eds.

\textbf{2002}. \emph{Cultures of Multiple Fathers: The theory and
practice of partible paternity in South America}. Gainesville,
University of Florida Press.

Bird-David, Nurit.

\textbf{1990}. "The Giving Environment: Another Perspective on the
Economics System of Hunter-gatherers.". In \emph{Current Anthropology}
31: 183-96.

Bonilla, Oiara.

\textbf{2005}. ``O bom patrão e o inimigo voraz: predação e comércio na
cosmologia paumari''. \emph{Mana}. \emph{Estudos de Antropologia
Social}, 11(1): 41-66.

Brightman, Marc.

\textbf{2010}. ``Creativity and control: property in Guianese
Amazonia''. \emph{Journal de la Societe des Americanistes de Paris},
96(1): 135-167.

Brightman, Robert.

\textbf{1996}. "The Sexual Division of Foraging Labor: Biologt, Taboo,
and Gender Politics". \emph{Comparative Studies in Society and History,
Vol. 38, No. 4}, pp. 687-729.

Brochado, J.P.

\textbf{1984.} \emph{An ecological model of the pread of pottery and
agriculture into Eastern South America}, ph.d.dissertation, University
of Illinois at Urbana-Champaign, Ann Arbor UMI, Inc.

Cabral, Joana.

\textbf{2012}. \emph{Entre plantas e palavras. Modos de constituição de
saberes entre os Wajãpi (AP)}. Tese de Doutorado, Universidade de São
Paulo.

Carneiro da Cunha, Manuela.

\textbf{1978}. \emph{Os mortos e os outros: uma análise do sistema
funerário e da noção de pessoa entre os índios Krahô}. São Paulo:
Hucitec.

Cesarino, Pedro.

\textbf{2008}. \emph{ONISKA: A poética da morte e do mundo entre os
Marubo da Amazônia ocidental}. Tese de Doutorado. Rio de Janeiro. PPGAS,
Museu Nacional.

\textbf{2010}. ``Donos e duplos: relações de conhe-cimento, propriedade
e autoria entre os Marubo''. \emph{Revista de Antropologia},
53(1):147-197.

CIMI- Conselho Indigenista Missionário.

\textbf{2015}. ``Ameaçados, Awá Guajá isolados aceitam contato no
Maranhão''. http://www.cimi.org.br/site/pt-
br/?system=news\&conteudo\_id=7946\&action=read\#, acesso em 05/05/2015.

Clastres, Hélène.

\textbf{1978 (1975)}. \emph{Terra sem mal: o profetismo tupi-guarani}.
São Paulo: Brasiliense.

Clastres, Pierre.

\textbf{1968}. ``Entre silêncio e dialogo''. In \emph{L'arc Documentos-
Lévi-Strauss}. São Paulo: Editora Documentos.

\textbf{1995} (1972). \emph{Crônica dos índios Guayaki: o que sabem os
Aché, caçadores nômades do Paraguai}. Rio de Janeiro: Editora 34.

\textbf{2003} (1974). \emph{A sociedade contra o Estado -- pesquisas de
antropologia política}. São Paulo: Cosac \& Naify.

\textbf{2004} (1976). ``A economia Primitiva''. In \emph{Arqueologia da
Violência. São Paulo: Cosac \& Naify}.

Coelho de Souza, Marcela.

\textbf{2004}. ``Parentes de sangue: incesto, substância e relação no
pensamento Timbira''. \emph{Mana}. \emph{Estudos de Antropologia
Social}, 10(1):25-60.

Collier, Jane \& Sylvia Yanagisako.

\textbf{1987}. ``Toward a Unified analysis of gender and kinship''. In:
Collier, Jane \& Sylvia Yanagisako (eds), \emph{Gender and Kinship.
Essays toward a Unified Analysis}. Standford, Standford University
Press.

Cormier, Loretta A.

\textbf{2000}. \emph{The ethnoprimatology of the Guajá indians of
Maranhão}, \emph{Brazil}. Doctor Thesis. Department of Anthropology -
Tulane University.

\textbf{2003}. \emph{Kinship With Monkeys: The Guajá foragers of eastern
Amazonia}. New York: Columbia University Press.

\textbf{2005}. "Um aroma no ar: a ecologia histórica das plantas
anti-fantasma entre os Vuajá da Amazônia". In: \emph{Mana}.
\emph{Estudos de Antropologia Social}, 11(1): 129-154.

Corsín Jimenez, Alberto; Willerslev, Rane.

\textbf{2007}. ``An anthropological concept of the concept':
reversibility among the Siberian Yukaghirs''. \emph{Journal of the Royal
Anthropological Institute} (N.S.), 13:527-544

Costa, Luiz Costa.

\textbf{2013}. ``Alimentação e comensalidade entre os Kanamari da
Amazônia Ocidental''. \emph{Mana- Estudos de Antropologia Social},
19(3): 473-504.

Cunha, Péricles.

\textbf{1987}. \emph{Análise fonêmica preliminar da língua Guajá}.
Dissertação de Mestrado. Universidade de Campinas.

Da Matta, Roberto.

\textbf{1973}. \emph{Ensaios de Antropologia Estrutural}. Petrópolis:
Vozes.

\textbf{1976}. \emph{Um mundo dividido: a estrutura social dos índios
Apinayé}. Petrópolis, Vozes.

Dal Poz Neto, João.

\textbf{2004}. \emph{Dádivas e Dívidas na Amazônia -- parentesco,
economia e ritual nos Cinta-Larga}. Tese de Doutorado. Campinas:
Unicamp.

Descola, Philippe.

\textbf{1986}. \emph{La nature domestique: symbolisme et praxis dans
l'écologie des Achuar}. Paris: Maison des Sciences de l'Homme.

\textbf{1991}. ``El determinismo raquítico''. In \emph{Etnoecológica}
vol.1 nº 1 (Abril).

\textbf{1992}. "Societies of nature and the nature of society", In: A.
Kuper (org.). \emph{Conceptualizing society}. Londres: Routledge,
pp.107-126.

\textbf{1993}. Les affinités sélectives: alliance, guerre et predation
dans l'ensemble jivaro", in: P. Descola \& A.-C. Taylor (orgs). \emph{La
remontée de l'Amazone: anthropologie et histoire des sociétes
amazoniennes. L'Hommme, 126-128}, pp.171-90.

\textbf{1996}. La Selva Culta: simbolismo y praxis en la ecologia del
los Ashuar. Quito: Abya-Yala.

\textbf{1998}. "Estrutura ou sentimento: a relação com o animal na
amazônia. \emph{In}: Mana. \emph{Estudos de Antropologia Social}, 4 (1):
23-45.

\textbf{2005}. \emph{Par-delà nature et culture}. Paris: Gallimard.

\textbf{2006 (1994)}. \emph{As Lanças do Crepúsculo: Relações jívaro na
Alta Amazônia}. São Paulo: Cosac Naify.

Descola, Philippe \& Gísli Pálsson.

\textbf{1996}. ``Introduction''. In Descola, Philippe \& Gísli Pálsson
(orgs.). \emph{Nature and Society: Anthropological Perspectives}.
London: Routledge.

Dodt, Gustavo.

\textbf{1939}. Descripção dos Rios Parnahyba e Gurupi.
\emph{Brasiliana}, Série 5\textsuperscript{{a}}, 138. São Paulo:
Companhia Editora Nacional.

Dooley, Robert A.

\textbf{1982}. \emph{Vocabulário do Guarani}. Brasília, DF: Summer
Institute of Linguistics.

Dumont, Louis.

\textbf{1953}. ``The Dravidian Kinship Terminology as Expression of
Marriage''. \emph{Man}, art. 54.

Erikson, Philippe.

\textbf{1987}. "De l'apprivoisement a l'approvisionnement: chasse,
alliance et familiarisation en Amazonie amérindienne: . In
\emph{Techniques et cultures}, 9: 105-140.

Fausto, Carlos.

\textbf{1991}. \emph{Os Parakanã: dravidianato e casamento avuncular na
Amazônia}. Museu Nacional, dissertação de mestrado.

\textbf{1995}. ``De primos e sobrinhas: terminologia e aliança entre os
Parakanã (Tupi) do Pará''. In \emph{Antropologia do Parentesco- Estudos
Ameríndios}. Viveiros de Castro, eduardo (org). Rio de Janeiro: Editora
UFRJ.

\textbf{2001}. \emph{Inimigos fiéis: história, guerra e xamanismo na
Amazônia}. São Paulo: EDUSP.

\textbf{2008}. Donos demais: maestria e domínio na Amazônia. Mana.
\emph{Estudos de Antropologia Social}, 14 (2) : 329-366.

Feld, Steven.

\textbf{1994.} ``From Ethnomusicology to Echo-Muse-Ecology: Reading R.
Murray Schafer in the Papua New Guinea Rainforest''. In The Soundscape
Newsletter, no 08, June.
http://www.acousticecology.org/writings/echomuseecology.html

Fernandes, Florestan.

\textbf{1963} (1949). \emph{A Organização Social dos Tupinambá.} São
Paulo: Difel.

Forline, Louis Carlos.

\textbf{1997}. \emph{The persistence and cultural transpormation of the
Guajá indians: foragers of Maranhão State}, \emph{Brazil}. Doctor
Thesis. University of Florida.

Gallois, Dominique Tilkin.

\textbf{1988}. \emph{O movimento na cosmologia Waiãpi: criação, expansão
e tranformação do universo}. Tese de Doutorado. PPGAS/USP. São Paulo.

\textbf{2007}. "Gêneses waiãpi, entre diversos e diferentes". In:
\emph{Revista de Antropologia}, v.50, nº1.

Forline, Louis Carlos \& Uirá F.Garcia.

\textbf{2006}. "Awá-Guajá: perspectivas para o novo milênio". In:
Ricardo, Fany \& C.A.Ricardo (orgs). \emph{Povos indígenas no Brasil}.
São Paulo: ISA.

Garcia, Uirá.

\textbf{2010}. \emph{Karawara: a caça e o mundo dos Awá-Guajá}. Tese de
Doutorado, Universidade de São Paulo.

\textbf{2012a}. ``Ka'á Watá, `andar na floresta': caça e território em
um grupo tupi da Amazônia''. \emph{Mediações -- Revista de Ciências
Sociais}, 17(1):172-190.

\textbf{2012b}. ``O funeral do caçador: caça e perigo na Amazônia''.
\emph{Anuário Antropológico}, 2011(II):33-55.

\textbf{2015}. ``Sobre o poder da criação: parentesco e outras relações
awá-guajá. \emph{Mana. Estudos de Antropologia Social}, v. 21, p.
91-122,

Gell, Alfred.

\textbf{1998}. \emph{Art and Agency: an Anthropological theory}. Oxford:
Claredon.

Goldman, Marcio.

\textbf{2006}. ``Alteridade e experiência: antropologia e teoria
etnográfica''. \emph{Etnográfica -- Revista do Centro de Estudos de
Antropologia Social}, v. 10, n. 1, p. 161-173.

Gomes, Mércio Pereira.

\textbf{1991}. ``O povo Guajá e as condições reais para sua
sobrevivência''. In \emph{Povos Indígenas no Brasil 1987/88/89/90}. São
Paulo: Centro Ecumênico de Documentação e Informação - CEDI.

Gonçalves, Marco Antonio.

\textbf{2001}. \emph{O mundo inacabado: ação e criação em uma cosmologia
amazônica}. Rio de Janeiro: Editora UFRJ.

Gow, Peter.

\textbf{1991}. \emph{Of Mixed Blood: kinship and history in Peruvian
Amazonia}. Oxford: Clarendon.

\textbf{1997}. "O Parentesco como consciência humana: o caso dos Piro".
In: Mana. \emph{Estudos de Antropologia Social}, 3(2); 39-65.

Grenand, Pierre.

\textbf{1982}. \emph{Ansi Parlaient nos Ancêtres: Essai d'Ethnohistoire
Waiãpi}. Paris, ORSTOM.

Haraway, Donna.

\textbf{2003}. \emph{The Companion Species Manifesto: dogs, people, and
significant otherness.} Chicago: Prickly Paradigm Press.

Havt, Nadja.

\textbf{2001}. \emph{Representações do ambiente e territorialidade entre
os Zo'e/ PA}. Dissertação de Mestrado. São Paulo: PPGAS/USP.

Hawkes, Kristen; Kim Hill and James F. O'Connell.

\textbf{1982}. ``Why Hunters Gather: Optimal Foraging and the Aché of
Eastern Paraguay''. In: \emph{American Ethnologist}. Vol. 9, No. 2,
Economic and Ecological Processes in Society and Culture (May, 1982),
pp. 379-398.

Heckenberger, Michael, Eduardo G.Neves \& James B.Petersen.

\textbf{1998}. "Como nascem os modelos? As origens e expansões Tupi na
Amazônia Central. In: Revista de Antropologia, v.41.

Hill, K., \& Hawkes, K.

\textbf{1983}. ``Neotropical hunting among the Ache of eastern
Paraguay''. In R. B. Hames, \& W. T. Vickers (Eds.), \emph{Adaptive
responses of native Amazonians}. (pp. 139-188). Academic Press.

Holmberg, Allan R.

\textbf{1969}. \emph{Nomads of the long bow -- the Siriono of Eastern
Bolivia}. New York: The American Museum of Natural History -- The
Natural History Press.

Hugh-Jones, Stephen.

\textbf{1996}. "Bonnes raisons ou mauvaise conscience? De l'ambivalence
de certains Amazoniens envers la consommation de viande". \emph{Terrain,
26}, pp.123-48.

Huxley, Francis.

\textbf{1963}. \emph{Selvagens Amáveis:um antropologista entre os índios
Urubus do Brasil}. Rio de Janeiro: Companhia Editora Nacional.

Ingold, Tim.

\textbf{1992}. ``Editorial''. In \emph{Man} - \emph{The Journal of the
Royal Anthropological Institute}, \emph{New Series}, vol. 27, nº 4 -
December.

\textbf{1996}. ``The optimal forager and economic man''. In Descola,
Philippe \&, Gísli Pálsson (orgs). \emph{Nature and Society}:
\emph{Anthropological Perspectives}. London: Routledge.

\textbf{2000}. \emph{The perception of the environment: essays in
livelihood, dwelling and skill}. London: Routledge.

\textbf{2003}. ``A Evolução da Sociedade''. In Fabian, A.C. (org).
\emph{Evolução, Sociedade, Ciência e Universo}. Bauru: Editora da
Universidade do Sagrado Coração.

\textbf{2007}. \emph{Lines: a brief history}. London: Routledge.

Ingold, Tim, David Riches \& James Woodburn (orgs).

\textbf{1988}. \emph{Hunters and gatherers 1: history, evolution and
social change}. Washington D.C: Berg.

\textbf{1988}. \emph{Hunters and gatherers 2: property, power and
ideology}. Washington D.C: Berg.

Jara, Fabíola.

\textbf{1996}. \emph{El camino del kumu: ecología y ritual entre los
Akuriyó de Surinam.} Quito: Abya-Yala.

Kohn, Eduardo.

\textbf{2007}. ``Animal masters and the ecological embedding of history
among the Ávila Runa of Ecuador''. In: Carlos Fausto \& Michael
Heckenberger (orgs.), \emph{Time and memory in indigenous Amazonia:
anthropological perspectives.} Gainesville: University Press of Florida.
pp. 106-129.

\textbf{2013}. \emph{How forest think. Toward na Anthropology beyond the
Human}. Berkeley/ Los Angeles: University of California Press.

Kopenawa, David \& Albert Bruce.

\textbf{2013}. \emph{The falling sky: Words of a Yanomami shaman}.
Cambridge, MA: Harvard University Press.

Kracke, Waud.

\textbf{1978}. \emph{Force and Persuasion: Leadership in na Amazonian
Society}. Chicago, The Universtity of Chicago Press.

Laraia, Roque.

\textbf{1986}. \emph{Tupi: Índios do Brasil Atual}. São Paulo, FFLCH,
Universidade de São Paulo.

Latour, Bruno.

\textbf{1994} (1991). \emph{Jamais Fomos Modernos: Ensaio de
Antropologia Simétrica.} São Paulo, Editora 34.

Lea, Vanesa.

\textbf{2012}. \emph{Riquezas intangíveis de pessoas partíveis: os
Mebêngôkre (Kayapó) do Brasil Central}. São Paulo: Edusp.

Leach, Edmund.

\textbf{1983}. \emph{Edmund Leach} (coletânea de artigos). Da Matta,
Roberto (org). São Paulo: Ática.

Lee, Richard \& Richard Daly, eds.

\textbf{1999}. \emph{The Cambridge encyclopedia of hunters and
gatherers}. Cambridge: Cambridge University Press.

Lévi-Strauss, Claude.

\textbf{1970 {[}1962{]}}. \emph{O pensamento selvagem}. São Paulo:
Companhia Editora

Nacional.

\textbf{1993}. \emph{História de Lince}. São Paulo: Cia. das Letras.

\textbf{2004 {[}1964{]}}. \emph{O cru e o cozido}. São Paulo: Cosac
Naify.

\textbf{2004 {[}1966{]}}. \emph{Do mel as cinzas}. São Paulo: Cosac
Naify.

Lima, Tânia Stolze.

\textbf{1995}. \emph{A parte do Cauim: etnografia Juruna}. Tese de
Doutorado. PPGAS -- Museu Nacional/UFRJ. Rio de Janeiro.

\textbf{1996}. ``O dois e seu múltiplo: reflexões sobre o perspectivismo
em uma cosmologia tupi''. In \emph{Mana}. \emph{Estudos de Antropologia
Social}, 2(2): 21-47.

\textbf{1999}. ``Para uma teoria etnográfica da distinção natureza e
cultura na cosmologia Juruna''. In \emph{Revista Brasileira de Ciências
Sociais. Vol 14 n° 40, junho}.

\textbf{2005}. \emph{Um peixe olhou para mim -- O povo Yudjá e a
Perspectiva}. São Paulo: ISA/ editora Unesp/ NuTI.

Luciani, José Antonio Kelly.

\textbf{2001}. ``Fractalidade e troca de perspectivas''. In: \emph{Mana.
Estudos de Antropologia Social,} 7(2):95-132. Rio de Janeiro.

Macedo, Valéria Mendonça.

\textbf{2010}. \emph{Nexos da Diferença: Cultura e afecção em uma aldeia
guarani na Serra do Mar}. Tese de Doutorado. São Paulo: Universidade de
São Paulo.

Magalhães, Marina Marina Silva.

\textbf{2002}. \emph{Aspectos Morfológicos e Morfossintáticos da Língua
Guajá}. Dissertação de Mestrado. Universidade de Brasília.

\textbf{2005}. ``Pronomes e Prefixos Pessoais do Guajá''. In Rodrigues,
Aryon Dall'Igna \& Ana Suelly Arruda C.C (orgs). \emph{Novos estudos
sobre línguas indígenas}. Brasília: Editora UNB.

\textbf{2007}. \emph{Sobre a Morfologia e a Sintaxe da Língua Guajá
(Família Tupi-Guarani)}. Tese de Doutorado. Brasília-DF: Universidade de
Brasília.

\textbf{2010}. \emph{Fala de Irakatakôa}. Manuscrito inédito.

\textbf{2013}. Levantamento da documentação existente sobre o povo
indígena Awá-Guajá e registro e sistematização de informações
sociolinguísticas e demográficas atuais. Relatório, CGIIRC-Funai/ GIZ,
Brasília.

Maizza, Fabiana.

\textbf{2014}. "Sobre as crianças-planta: o cuidar e o seduzir no
parentesco Jarawara". In: \emph{Mana. Estudos de Antropologia Social}
20(3):491-518.

Martins, Marlúcia Bonifácio \& Tadeu Gomes de Oliveira (orgs).

\textbf{2011}. \emph{Amazônia Maranhense: diversidade e conservação}.
Belém: Museu Paraense Emílio Goeldi.

Müller, Regina Pólo.

\textbf{1990}. Os Asuriní do Xingu: história e arte. Campinas:
Ed.Unicamp.

Nimuendajú, Curt.

\textbf{1948}. ``The Guajá, by Curt Nimuendajú''. In Steward, Julian
Haynes (org). \emph{Handbook of South American Indians-} \emph{v 3}.
Washington: Govt. Print. Off.

\textbf{1987 (1914)}. \emph{As lendas da criação e destruição do mundo
como fundamentos da religião dos Apapocúva-Guarani}. São Paulo:
Hucitec/Edusp.

Neves, Eduardo Góes.

\textbf{1999}. ``O Velho e o Novo na Arqueologia Amazônica''. Revista
USP, Brasil, v. 44, p. 87-113.

Noelli, F.S.

\textbf{1996}. "As hipóteses sobre o centro de origem e rotas de
expansão dos Tupi", \emph{Revista de Antropologia} 39(2):7-53.

O'dwyer, Eliane Cantarino.

\textbf{2001}. \emph{Laudo antropológico- área indígena Awá}. Fundação
Nacional do Índio (Funai).

\textbf{2010}. \emph{O papel social do Antropólogo}. Rio de Janeiro:
Laced/ e-papers.

Overing, Joanna.

\textbf{2006}. "O Fétido odor da morte e os aromas da vida. Poética dos
saberes e processo sensorial entre os Piaroa da Bacia do Orinoco". In:
Revista de Antropologia, V. 49 No 1.

Overing-Kaplan, Joanna.

\textbf{1975}. \emph{The Piaora- a People of the Orinoco Basin: A Study
in Kinship and Marriage}. Oxford: Claredon Press.

Pissolato, Elisabeth.

\textbf{2007}. \emph{A Duração da Pessoa: mobilidade, parentesco e
xamanismo mbya (guarani)}. São Paulo: Editora Unesp.

Ramos, Alcida R.

\textbf{1990}. "Ethnology Brazilian Style". \emph{Cultural Anthropology}
5, no. 4, pp. 452-472.

Ribeiro, Darcy.

\textbf{1980}. \emph{Uirá sai à procura de Deus: Ensaios de Etnologia e
Indigenismo}. Rio de Janeiro: Paz \& Terra.

\textbf{1996}. \emph{Diários Índios -- Os Urubu-Kaapor}. São Paulo:
Companhia das Letras.

Rival, Laura M.

\textbf{1993}. ``The Growth of Family Tress: Huaorani Conceptualization
of Nature and Society.'' \emph{Man} 28(4): 635-52.

\textbf{1998}. "Androgynous parents and guest children: the Huaorani
couvade". \emph{Journal of the Royal Anthropological Institute, 4} (4),
pp.619-42.

\textbf{1999}. "Introduction: South America. In: Lee, Richard \& Richard
Daly, \emph{The Cambridge encyclopedia of hunters and gatherers}.
Cambridge: Cambridge University Press.

\textbf{2002}. \emph{Trekking Through History - the Huaorani of
Amazonian Ecuador}. New York: Columbia University Press.

Rivière, Peter.

\textbf{1969}. \emph{Marriage Among The Trio: A Principle of Social
Organization}. Oxford: Clarendon Press.

Rodrigues, Aryon Dall'Igna.

\textbf{1984-85}. ``Relações internas na família linguística Tupi
Guarani''. São Paulo: \emph{Revista de Antropologia 27-28}.

Roosevelt, Anna C.R.

\textbf{1992}. ``Ancient and Modern Hunter-Gatherers of Lowland South
America: An Evolutionary Problem''. In Balée, William (org).
\emph{Principles of Historical Ecology}: 190-212. New York: Columbia
University Press.

Rosalen, Juliana.

\textbf{2005}. Aproximação à temática das DST junto aos Wajãpi do
Amapari. Um estudo sobre malefícios, fluidos corporais e sexualidade.
Dissertação de Mestrado. Universidade de São Paulo.

Sahlins, Marshall.

\textbf{1997a}. ``O `pessimismo sentimental' e a experiência
etnográfica: por que a cultura não é um "objeto" em via de extinção
(parte I)". \emph{Mana}. \emph{Estudos de Antropologia Social,} v.3,
n.1, abril.

\textbf{1997b}. ``O `pessimismo sentimental' e a experiência
etnográfica: por que a cultura não é um "objeto" em via de extinção
(parte II). \emph{Mana}. \emph{Estudos de Antropologia Social,} vol.3,
no.2, p.103-150, outubro.

\textbf{2011}. ``What kinship is (part one)''. JRAI (NS), 17:2-19;
``What kinship is (part two)'', JRAI (NS), 17:227-42.

Santos, Rosana de Jesus Diniz.

\textbf{2015}. \emph{Awa Papejapoha: um estudo sobre educação escolar
entre os awá guajá/MA}. Monografia de especialização em Desenvolvimento
e relações sociais no campo: diversidade e interculturalidade dos povos
originários, comunidades tradicionais e camponesas do Brasil,
Universidade de Brasília.

Schneider, David M.

\textbf{1968}. \emph{American kinship: a cultural account}. Chicago: The
University of Chicago Press.

Schroeder, Ivo.

\textbf{2006}. \emph{Política e parentesco nos} Xerente. Tese de
doutorado, Universidade de São Paulo.

Seeger, Anthony.

\textbf{1981}. \emph{Nature and Society in Central Brazil: The Suyá
Indians of Mato Grosso}. Cambridge, Mass.: Harvard University Press.

Seeger, Anthony, Roberto A. da Matta \& Eduardo viveiros de Castro.

\textbf{1979}. "A construção da pessoa nas sociedades indígenas
brasileiras". \emph{Boletim do Museu Nacional, 32}, pp.2-19.

Service, Elman R. \emph{Os caçadores}. Rio de Janeiro: Zahar Editores,
1971 (1966).

Sigrist, Tomas.

\textbf{2008}. \emph{Guia de Campo}: aves da Amazônia Brasileira. São
Paulo: Avisbrasilis.

Silva, Márcio Ferreira.

\textbf{1995a}. ``Sistemas Dravidianos Amazônicos: o caso
waimiri-atroari''. In Viveiros de Castro, Eduardo (org).
\emph{Antropologia do Parentesco -- Estudos Ameríndios.} Rio de Janeiro:
Editora UFRJ.

\textbf{2005}. \emph{Sistemas de aliança na América do Sul Tropical:
modelos e práticas}. Manuscrito (inédito).

Silverwood-Cope, Peter L.

\textbf{1990}. \emph{Os Makú: povo caçador do noroeste da Amazônia}.
Brasília: Editora UnB.

Sponsel, Leslie E.

\textbf{1997}, "The Human Niche in Amazonia : Explorations in
Ethnoprimatology". In Warren G. Kinzey (org). \emph{New World Primates:
Ecology, Evolution, and Behavior}. New York, NY: Aldine de Gruyter, pp.
143-165.

Stearman, Allyn MacLean.

\textbf{2001 (1989)}. \emph{Yuquí: Forest Nomads ina a Changing World.}
San Francisco: Holt, Rinehart \& Winston.

\textbf{1992}. ``Neotropical Indigenous Hunters and tehir neighbors.
Sirionó, Chimane, and yuquí Hunting on the Bolivian Frontier''. In
Redford, Kent H \& Christine Padoch (orgs). \emph{Conservation of
neotropical forests: working from traditional resource use}. New York:
Columbia University Press.

Strathern, Marilyn.

\textbf{1980}. ``No Nature, no culture: the Hagen case. In Nature,
Culture, and Gender. C.MacCormack and M.Strathern, eds.Pp-174-222.
Cambridge: University of Cambridge Press.

\textbf{1988}. \emph{The Gender of the Gift: Problems with Women and
Problems with Society in Melanesia}. Berkeley: University of California
Press.

\textbf{1995a}. \emph{The relation: issues in complexity and scale}.
Cambridge: Prickly Pear Press.

\textbf{1995b}. ``Necessidade de Pais, Necessidade de Mães''. In Revista
de Estudos Feministas, vol 3 nº 2.

\textbf{1999a}. \emph{Property, substance and effect: anthropological
essays on persons and things}. London: The Athlone Press.

\textbf{1999b}. ``No limite de uma certa linguagem (entrevista)''. In
\emph{Mana}. \emph{Estudos de Antropologia Social} 5(2): 157-175.

\textbf{2006 (1988)}. \emph{O Gênero da Dádiva} - \emph{Problemas com as
mulheres e problemas com a sociedade na Melanésia}. Campinas: Editora da
Unicamp.

\textbf{2014a}. ``From Papua New Guinea to a UK Council on Bioethics:
fieldwork at the beginning and end of an anthropological lifetime''.
Mimeo, Opening lecture for XII GEC (`students in the field') conference,
São Paulo, USP.

\textbf{2014b.} Reading relations backwards. \emph{The Journal of the
Royal Anthropological Institute (N.S.) 20, 3-19}.

Sztutman, Renato.

\textbf{2012}. \emph{O profeta e o principal}. São Paulo: Edusp.

Taylor, Anne-Christine.

\textbf{1993}. "Remembering to forget: identity, mourning and memory
among the Jívaro". \emph{Man, 28} (4), pp.653-78.

\textbf{1996}. "The Soul's Body and it's States: An Amazonian
Perspective on the Nature of Being Human". In: \emph{The} \emph{Journal}
\emph{of the Royal Anthropological Institute}, Vol 2, nº2, 201-215.

\textbf{2001}. ``Wives, Pets and Affines: Mariage among Jivaro''. In
Rival, Laura \& Neil L. Whitehead (orgs). \emph{Beyond the Visible and
the Material- the amerindization of society in the work of Peter
Rivière.} Oxford: Oxford University Press.

Toral, André.

\textbf{2007}. ``Caminhando só: Comentários sobre o filme Serra da
desordem (2006), de Andréa Tonacci". In: \emph{Facom}, nº 17.

Trautmann, T. R \& Barnes R.H.

\textbf{1998}. ``Dravidiam, Iroquis ans Crow-Omaha in North American
Perspective'' In: Godelier, M. Trautmann, Tjon Sie Fat, F. E.
Transformations of Kinship.Washington: Smithsonian Institution Press.

Vander Velden, Felipe Ferreira

\textbf{2010}. \emph{Inquietas Companhias: sobre os animais de criação
entre os Karitiana}. Tese de Doutorado. Campinas: Universidade de
Campinas.

Vaz, Antenor.

\textbf{2011}. ``Isolados no Brasil -- Políticas de Estado: Da tutela
para a Política de Direitos -- Uma Questão Resolvida?''. In: Informe 10,
IWGIA. Brasília: Grupo Internacional de Trabalho sobre Assuntos
Indígenas.

\textbf{2014}. ``Povos Indígenas Isolados e de Recente Contato no Brasil
-- A que será que se destinam?''. In: \emph{Le monde Diplomatique} --
Brasil. Agosto 2014.

Viveiros de Castro, Eduardo.

1984/85. ``Bibliografia etnológica básica tupi-guarani''. In
\emph{Revista de Antropologia}, vol. 27/28. São Paulo: USP.

1986. \emph{Araweté: os deuses canibais}. Rio de Janeiro: Ed. Jorge
Zahar.

\textbf{1992}. \emph{From the Enemies Point of View: Humanity and
Divinity in an Amazonian Society}. Chicago: The University of Chicago
Press.

\textbf{1993}. "Alguns aspectos da afinidade no dravidianato amazônico",
In: E.Viveiros de Castro \& M. Carneiro da Cunha (orgs). \emph{Amazônia:
etnologia e história indígena}. são Paulo: NHII-USP/ FAPESP,
pp.150-2010.

\textbf{1996}. ``Os pronomes cosmológicos e o perspectivismo ameríndio.
In \emph{Mana}. \emph{Estudos de Antropologia Social}, 2(2):115-144.

\textbf{2002}. \emph{A inconstância da alma selvagem- e outros ensaios
de antropologia}. São Paulo: Cosac \& Naify.

\textbf{2007}. "Filiação Intensiva e Aliança Demoníaca". In:
\emph{Revista Novos Estudos, Cebrap}.

\textbf{2008}. "Xamanismo transversal: Lévi-Strauss e a cosmopolítica
amazônica". In: Caixeta de Queiroz, Ruben \& Renarde Freire Nobre,
(Orgs.). \emph{Lévi-Strauss: leituras brasileiras}. Belo Horizonte:
Editora UFMG.

\textbf{2009}. : The Gift and the Given: three nano-essays on kinship
and magic". In S.Bamford \& J.Leach (orgs.), \emph{Kinship and Beyond:
the genealogical model reconsidered}. Oxford: Bergham Books.

Wagley, Charles.

\textbf{1988 (1977)}. \emph{Lágrimas de Boas Vindas: os índios Tapirapé
do Brasil Central}. São Paulo: Itatiaia/USP.

Wagley, Charles \& Eduardo Galvão.

\textbf{1961}. \emph{Os índios Tenetehara: uma cultura em transição}.
Rio de Janeiro: MEC, 1961.

Wagner, Roy.

\textbf{2010 (1981)}. \emph{A invenção da cultura}. São Paulo, Cosac
Naify.

Willerslev, Rane.

\textbf{2007}. \emph{Soul Hunters: hunting, animism, and personhood
among the siberian yukaghirs}. Berkeley, Los Angeles, London: University
of California Press.

Yokoi, Marcelo.

\textbf{2014}. \emph{Na Terra, no céu: os Awá-Guajá e os Outros}.
Dissertação de Mestrado. São Carlos: Universidade Federal de São Carlos.

\textbf{Trabalhos acadêmicos referentes os Awá-Guajá consultados }

Balée, William. 1987. \emph{Culturas da floresta: apontamentos críticos
sobre a ecologia cultural na Amazônia}, trabalho lido no Simpósio
ABA/ANPOCS.

Balée, William. 1992. ``O povo da capoeira velha: caçadores-coletores
das terras baixas da América do Sul''. Trabalho apresentado na
Conferência Amazônica da Fundação Memorial da América Latina em 25 de
Março.

Balée, William. 1992. ``People of the Fallow: A historical Ecology of
Foraging in Lowland South América''. In Redford, Kent H \& Christine
Padoch (orgs). \emph{Conservation of neotropical forests: working from
traditional resource use}. New York: Columbia University Press.

Balée, William. 1994. \emph{Footprints of the forest: Ka'apor
ethnobotany -- the historical ecology of plant utilization by an
amazonian people}. New York: Columbia University Press.

Balée, William. 1997. ``Language, law, and land in Pre-Amazonian
Brazil''. In \emph{Texas International Law Journal} vol. 32 (no1,
winter).

Balée, William. 2000. ``Antiquity of traditional ethnobiological
knowledge in Amazonia: the Tupí-Guaraní family and time''. In
\emph{American Society for Ethnohistory} - Tulane University (Spring).

Beghin, François-Xavier. 1950. \emph{Les Guajá}. Letter to Darcy
Ribeiro. 15 July.

Beghin, François-Xavier. 1951. "Les Guajá". \emph{Revista do Museu
Paulista, N.S. 5:137-39}

Beghin, François-Xavier. "Relation du premier contact avec les indiens
Guajá". \emph{Journal de la Société des Américanistes,} N.S. XLVI:
197:204.

Cardoso, Guilherme Ramos. 2013. \emph{Uma leitura sobre identidade e
etnicidade na literatura sobre os Awá-Guajá}. Dissertação de mestrado.
Universidade Federal Fluminense.

Cormier, Loretta. 2003. \emph{Kinship With Monkeys}. New York: Columbia
University Press.

Cormier, Loretta. 2005. "Um aroma no ar: a ecologia histórica das
plantas anti-fantasma entre os Guajá da Amazônia". In: \emph{Mana}.
\emph{Estudos de Antropologia Social,} 11(1): 129-154.

Cunha, Péricles. 1987. \emph{Análise fonêmica preliminar da língua
Guajá}. Dissertação de Mestrado. Universidade de Campinas.

Forline, Louis Carlos. 1997. The persistence and cultural transformation
of the Guaja Indians: Foragers of Maranhao state, Brazil. Tese de
doutorado. University of Florida, Gainesville.

Forline, Louis Carlos. 2007. \emph{Por uma Síntese Biocultural:
relatório preliminar sobre as comunidades Guajá dos Postos Indígenas
awá, Tiracambu, Juriti e Guajá.} Reno: Universidade de Nevada.

Forline, Louis Carlos. 2011. "The body social and the body private: fine
tuning our understanding of partible paternity and reproductive
strategies among amazonian indigenous groups". In: \emph{Proceedings of
the Southwestern Anthropological Association,} 2011, Vol. 5, Pp. 1-8.

Hernando, Almudena, Gustavo Politis, Alfredo González Ruibal, Elizabeth
Beserra Coelho. 2011. "Gender, Power, and Mobility among the Awá-Guajá
(Maranhão, Brazil)". In: \emph{Journal of Anthropological Research,}
vol. 67, 2011.

Hernando, Almudena e Coelho, Elisabeth Maria Beserra (org). 2013.
\emph{Estudos sobre os Awá -- caçadores-coletores em transição.} São
Luís: Edufma.

Magalhães, Marina Silva. 2002. \emph{Aspectos Morfológicos e
Morfossintáticos da Língua Guajá}. Dissertação de Mestrado. Universidade
de Brasília.

Magalhães, Marina Silva. 2005. ``Pronomes e Prefixos Pessoais do
Guajá''. In Rodrigues, Aryon Dall'Igna \& Ana Suelly Arruda C.C (orgs).
\emph{Novos estudos sobre línguas indígenas}. Brasília: Editora UNB.

Magalhães, Marina Silva. 2007. \emph{Sobre a Morfologia e a Sintaxe da
Língua Guajá (Família Tupi-Guarani)}. Tese de Doutorado. Universidade de
Brasília.

Magalhães, Marina Silva. 2010. \emph{Fala de Irakatakôa}. Manuscrito
inédito.

Nimuendajú, Curt. 1949.The Guajá. In: \emph{Handbook of South American
Indians.} J. Steward, ed.Vol. 3. Washington, D.C.: U. S. Government
Printing Office. p. 135-36.

Oliveira, Silviene Fabiana de, et al. 1998. "The Awá-Guajá Indians of
the Brazilian Amazon. Demographic Data, sérum Protein Markers and Blood
Groups". In: Hum Hered 1998; 48: 163-168.

O'Dweyer, Eliane Cantarino. 2001. \emph{Laudo antropológico- área
indígena Awá}. Fundação Nacional do Índio (Funai).

O'Dweyer, Eliane Cantarino. 2010. \emph{O papel social do Antropólogo}.
Rio de Janeiro: Laced/ e-papers.

Prado, Helbert Medeiros. \emph{O impacto da caça versus a conservação de
primatas numa comunidade indígena Guajá}. Dissertação de Mestrado.
Universidade de São Paulo.

Prado, Helbert Medeiros, Louis Carlos Forline, Renato Kipnis. 2012.
"Hunting Practices among the Awá-Guajá: Towards a long-term analysis of
sustainability in an Amazonian indigenous community. In: \emph{Bol. Mus.
Para. Emilio Goeldi. Cienc.Hum,. Belém, v.7, n.2, p: 479-491}. mai-ago,
2012.

Rubial, Alfredo González, Almudena Hernando, Gustavo Politis. 2010.
"Ontology of the self and material culture: Arrow-making among the Awá
hunter-gatherers (Brazil)". In:\emph{Journal of Anthropological
Archeology}.

Rubial, Alfredo González, Gustavo Politis, Almudena Hernando, Elizabeth
Beserra de Coelho. 2010. "Domestic Space and Cultural Transformation
Among the Awá of Eastern Amazonia". In: \emph{BAR International Series
2183}/ \emph{Archaeological Invisibility and Forgotten Knowledge:
Conference Proceedings, Łódź, Poland, 5th--7th September 2007}.

\textbf{Documentos oficiais consultados}

\textbf{1961}. Decreto nº 51.026 de 25/7/61, Cria a Reserva Florestal do
Gurupi e dá outras providências. Jânio Quadros (Presidente do Brasil).

\textbf{1973}. Funai - \emph{Relatório da viagem ao alto rio Carú e
igarapé da Fome para verificar a presença de índios Guajá}. Valéria
Parise. São Luís.

\textbf{1973}. Funai - \emph{Relatório de viagem -- frente de atração
Guajá 6ªDR do Maranhão}. João Fernandes Moreira.

\textbf{1973}. Funai - \emph{Proposta para Interdição da Área Awá/Guajá
4ª SUER Maranhão.} Fiorello Parise.

\textbf{1980}. \emph{Parecer sobre o processo Funai/ BSB/ 5044/ 79,
referente aos índios Guajá do Estado do Maranhão}. Mércio Pereira Gomes.

\textbf{1980}. \emph{Relatório sobre o contato e a necessidade de
transferência de 27 índios Guajá do Igarapé Timbira, município de santa
Luzia, para a Reserva Caru, Município de Bom Jardim, estado do Maranhão.
E da necessidade de se criar um novo posto para esses e outros Guajá que
se encontram na Reserva Caru, e da Implementação de uma Política
indigenista própria para esses índios para que se possa evitar seu
declínio populacional}. Mércio Pereira Gomes.

\textbf{1980}. Centro Ecumênico de Documentação e Informação (CEDI) --
\emph{Ficha sobre a situação atual das populações indígenas no Brasil.}
Carlos Ubbiali.

\textbf{1980.} \emph{Carta de Mércio Pereira Gomes ao CEDI (Beto Ricardo
e Fany Ricardo}).

\textbf{1981}. Funai -- \emph{Relatório Preliminar sobre a Situação de
Grupos Guajá que se encontram fora de Reservas Indígenas e que precisam
de uma solução em caráter de urgência}. Mércio Pereira Gomes. São Luís.

\textbf{1982}. Funai -- \emph{A problemática Indígena no Maranhão,
especificamente nas áreas de influência imediata da ferrovia Carajás:
reserva Turiaçu, reserva Caru e reserva Pindaré}. Mércio Pereira Gomes.

\textbf{1982}. Comissão Pró-Índio do Maranhão -- \emph{Nota de Repúdio.}
Elizabeth Maria Beserra de Coelho.

\textbf{1982}. \emph{Resposta ao Del. da 6ª DR-Funai com acusações em
relação ao tratamento que a Funai dispensa aos Guajáe outros povos --
MA}. Mércio Pereira Gomes.

\textbf{1983}. Funai - \emph{Relatório de viagem da frente de atração
Guajá no período de 03/06/82 e 12/06/82}. Fiorello Parise. Funai.

\textbf{1983}. Funai - \emph{Relatório anual 1983 -- Postos de
Vigilância/ Frente de Atração Guajá}. Cornélio Vieira de Oliveira.

\textbf{1984}. Funai -- \emph{Relatório sobre debelação de invasão}.
Chefe de posto Guajá e Alto Turiaçú.

\textbf{1984}. Funai - \emph{Relatório sobre a Frente de Atração}. José
Araújo Filho.

\textbf{1984}. \emph{Carta ao Exmo. Senhor Dr.Nelson Marabuto,
Presidente da Funai}/ \emph{Programa Awá}. José Porfírio Fontenele de
Carvalho.

\textbf{1985}. \emph{Certificado do cartório de Carutapera}. Januário de
Sena Loureiro.

\textbf{1985}. Funai -- \emph{Memorial Descritivo de delimitaçãoo --
Área Indígena Awá.} Autor desconhecido.

\textbf{1985}. Funai/CVRD - \emph{Relatório sobre os índios Guajá
Próximos à ferrovia Carajás - Km 400}. Mércio Pereira Gomes.

\textbf{1985}. Funai -- \emph{Relatório do reconhecimento da área da
Serra da desordem}. José Carlos dos Reis Meirelles Júnior.

\textbf{1985}. Funai -- \emph{Relatório de viagem}. Sérgio de Campos
(Eng.Agrimensor).

\textbf{1985}. Funai/CVRD -- \emph{Programa Awá -- relatório Inicial}.
Mércio Pereira Gomes.

\textbf{1985}. Funai/CVRD -- \emph{Relatório antropológico sobre a área
indígena Guajá (Awá-Gurupi)}. Mércio Pereira Gomes.

\textbf{1985}. Funai/ CVRD - \emph{Área indígena Awá-Gurupi. Estudos e
proposta.} José Carlos Meirelles Júnior e Mércio Pereira Gomes.

\textbf{1985}. CVRD -- \emph{Nota para a imprensa.}

\textbf{1987}. Funai -- \emph{Parecer nº 171/87 -- GT.Interministerial
-- DEC. nº 94.945/87. Área Indígena Awá.} Romero Jucá Filho \emph{et
al}.

\textbf{1987}. Funai - \emph{Descrição do Perímetro. Anexo Área Indígena
Awá/Guajá}. Fiorello Parise.

\textbf{1987}. CIMI -- \emph{Parecer sobre os Awá} . Paulo Mahado
Guimarães.

\textbf{1987}. SUER/ SPAG -- Convênio Funai/CVRD. \emph{Relatório de
atividades de agosto, setembro e outubro de 1987}. Fiorello Parise.

\textbf{1987}. Funai - \emph{Carta referente a criação do Sistema de
Proteção}. Fiorello Parise.

\textbf{1987}. Funai -- \emph{Relatório de atividades junho/ julho de
1987}. Fiorello Parise.

\textbf{1987}. Funai - \emph{Relatório de atividades de junho a dezembro
de 1987}. Domingos Faria Pereira.

\textbf{1987.} Funai -- \emph{Área indígena Awá (Declaração de Ocupação
Indígena)}, ref: PROC. Funai/BSB/2582/85. Itagiba Christiano de O.C.
Filho.

\textbf{1987}. Funai -- \emph{Portaria nº 3.767, de 13 de novembro de
1987.} Romero Jucá Filho.

\textbf{1987}. Funai -- \emph{Memorial descritivo de delimitação, Área
Indígena Awá}. Cornélio Oliveira, Artur N. Mendes e Reinaldo Florindo.

\textbf{1987}. CEDI -- \emph{Campanha Guajá}.

\textbf{1988}. Funai -- \emph{Parecer nº 197/88 -- GT Interministerial
-- DEC.nº. 94.945/87}. Romero Jucá Filho \emph{et al}.

\textbf{1988}. \emph{Portaria Interministerial nº 076} de 03 de maio de
1988, de posse permanente da área Awá. João Alves Filho; Jader
Fontenelle Barbalho.

\textbf{1988}. \emph{Portaria Interministerial nº 158, de 08 de setembro
de 1988}.

\textbf{1988}. Funai -- \emph{Carta de Santa Inês}. Sidney Possuelo, et
al.

\textbf{1988}. Funai -- \emph{Relatório de atividades do programa
Awá-Guajá, convênio Funai/ CVRD, novembro, dezembro/ 87 e considerações
finais}. Fiorello Parise.

\textbf{1988}. Funai -- \emph{Relatório de atividades meses janeiro e
fevereiro/ 88}. Fiorello Parise.

\textbf{1988}. Funai -- \emph{Relatório de atividades do Convênio
Funai/CVRD, referente ao 1ºtrimestre de 1988}. Idalércio de Andrade
Moreira.

\textbf{1988}. Funai - \emph{Relatório de viagem às ADR's do Estado do
Maranhão}. Leda Aparecida Câmara de Azevedo \& Marta Luciana de Sá Pinto
Barbosa.

\textbf{1988.} Funai -- \emph{Relatório de atividades meses janeiro e
fevereiro/ 88}. Fiorello Parise.

\textbf{1988}. Funai -- \emph{Assessoria de índios isolados da 4ª Suer
-- junho 88}. Fiorello Parise.

\textbf{1988}. Funai - \emph{Relatório de Atividades da Assessoria de
Índios Isolados}. Dinarte Nobre de Madeiro.

\textbf{1988}. Funai -- \emph{Relatório de atividades da Assessoria de
Índios Isolados da 4ª SUER referente ao mês de junho}. Fiorello Parise.

\textbf{1988}. Funai -- \emph{Relatório de Atividades do Sistema de
Proteção Awá-Guajá}. Dimas Valente.

\textbf{1988}. Funai -- \emph{Relatório final de atividades de 1988, e
panorama sobre a situação da área de índios isolados da 4ª SUER}.
Fiorello Parise.

\textbf{1988}. Funai - \emph{Relatório} \emph{de atividades do mês de
outubro/88 do sistema de proteção Awá/Guajá}. Dimas Valente.

\textbf{1988}. Funai -- \emph{Comunicação -- índios Guajá sem contato na
sede do PINC}. Sabino Francisco Conceição Neto.

\textbf{1989}. Funai -- \emph{Relatório sobre aspectos gerais do PIN
Guajá e algumas sugestões de trabalho}. Egipson Nunes Correia.

\textbf{1989}. Funai - \emph{Parecer sobre o relatório sobre aspectos
gerais do PIN Guajá e algumas sugestões de trabalho}. Samuel Vieira
Cruz.

\textbf{1989}. Funai -- \emph{Relatório de levantamento da área indígena
Caru e Alto Turiaçu.} Fiorello Parise \emph{et al}.

\textbf{1989}. Funai -- \emph{Situação do Sistema de Proteção
Awá/Guajá}. Samuel Vieira Cruz.

\textbf{1989}. Funai - \emph{Viagem à Terra Sem Lei -- relatório de
levantamento da AI Awá: equipes nº 02 e 03}. Fiorello Parise.

\textbf{1989}. Funai - \emph{Proposta de trabalho para a vigilância da
área awá}. Samuel Vieira Cruz.

\textbf{1990}. Funai -- Viagem ao Sistema de Proteção Awa/Guajá.
Fiorello Parise.

\textbf{1990}. Funai -- \emph{Apresentação de relatório, PIN Guajá}.
Egipson Nunes Correia.

\textbf{1990}. Convênio CVRD -- Funai. Missão do Banco Mundial -- PAPP/
MA. Cornélio Vieira de Oliveira.

\textbf{1990}. Funai -- \emph{Informações sobre o grupo Miri-Miri, da
área Awá-Guajá}. Samuel Vieira Cruz.

\textbf{1990}. Funai -- \emph{Relatório de atividades do sistema de
proteção Awá/Guajá no exercício de 1990.} Renildo Matos dos Santos.

\textbf{1999}. \emph{Relatório Final- Assessoria Antropológica junto aos
Awá-Guajá}. Funai/ CVRD. Nadja Havt Bindá.

\textbf{2000}. \emph{Assessoria Antropológica - Relatório Final}. Funai/
CVRD. Nadja Havt Bindá.

\textbf{2001}. \emph{Laudo antropológico}. Processo judicial no
95.353-8. 5a Vara da Justiça Federal do Maranhão. Eliane Cantarino
O'Dwyer.

\textbf{2001}. \emph{Diagnóstico sócio-econômico e geoambiental das
Terras Indígenas do Pindaré, caru, Awá e Alto Turiaçu}. Elizabeth Maria
Beserra Coelho, Rogério Tavares Pinto e Kátia Núbia Ferreira Correa.

\textbf{2002}. Funai -- \emph{Relatório Awá-Guajá 2002}. Mércio Pererira
Gomes \& José Carlos Meireles. Petrópolis.

\textbf{2002}. Universidade federal de são Paulo/ Escola Paulista de
Medicina. \emph{Descrição das precariedades atuais relativas à saúde e
necessidades dos índios Guajá}. João Paulo Botelho Vieira Filho.

\textbf{2006}. Funai -- \emph{Relatório de Atividades Awá-Guajá, maio de
2006}. Maurício de Lima Wilke.

\textbf{2009}. Expedição da Funai confirma a existência de índios
isolados no Maranhão. Site da Funai,
http://www.funai.gov.br/index.php/comunicacao/noticias/2487-expedicao-da-funai-confirma-a-
existencia-de-indios-isolados-no-maranhao - acesso em 01/06/2015.

\textbf{2013}. Funai - Relatório de Incursão à Campo -- Programa Awá
Guajá. Nelson Cesar Destro Junior.

\textbf{2013}. Funai - Relatório sobre viagem para a TI Awá. Felipe V.M.
Almeida.

\textbf{S/d}. Funai - \emph{Relatório sobre a Frente de Atração
Awá-Guajá}. Domingos Faria Pereira.

\textbf{S/d}. Funai -- \emph{Carta a Sidney Possuelo}. Domingos Faria
Pereira.

\textbf{Filmografia}

Tonacci, Andrea.

\textbf{2006}. \emph{Serras da Desordem}. Longa Metragem.
