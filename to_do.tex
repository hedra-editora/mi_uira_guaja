To do:

1.Tirar numeração do sumário
2.Tirar aspas e itálico de longas citações
3.Padronizar citações (Nome autor, ano, p.)
4.Colocar imagens que são referenciadas no texto
5.Aspas iniciais estão invertidas (substituir " por ``)
6.Está saindo hífen ao invés de traço em grande parte do texto (substituir - por --)
7.Padronizar as palavras: Zo'é, Awá, Guajá, Ka'apor, Tenetehara, Carú, Turiwára, Funai, Assurini, Tiracambu, Gurupi
8.Todas as palavras dos títulos estão em caixa-alta, manter apenas a primeira em c.a.
9.Deixar em caixa-baixa as palavras rio, igarapé, bacia
10.Tirar do itálico as palavras Awá, Juriti, Tiracambu
11. Versalete: USP, TI, ONG, CIMI, UnB, séculos (XVII, XVIII, XIX, XX), PIN, ICMB, PGC, CVRD, EFC, PBA, REBIO, SPI, FPEAG 
12.Mudar dados da ficha técnica (ISBN, revisão, agradecimentos)
13.Italizar "apud" e colocar ., depois da expressão
14. Colocar em caixa-alta as expressões: Terra(s) Indígena(s), Frente de Atração
15. Palavras com caracter que não puxou do word: Juriximatya, Amy Pirahy, Muturuhu, Panypinuhu, Akamatya, Xiramuku, haxy, haty

====================================

PRÓLOGO: foto l. 173
		 
IMAGENS: Cap3. Foto l. 2182 - (foto ximixia) diminuir um pouco para caber na página.
		 Cap6. Foto l. 2017 (do quati sendo mexido) - idem
		 Cap7. Foto l. 311 - idem
		 Cap7. Foto l. 1425 - idem
		 
TABELAS: Verificar itálicos e símbolos

==================================================
UIRÁ:

A seguir te informo sobre a fotos de acordo com o nome do arquivo que elas tem na pasta.

1- As nove fotos, correspondentes a abertura de cada capítulo são as seguintes (segue o nome dos arquivos).

Cap 1 - “Irakatakoa_426” (Domenico Pugliese)

Cap 2 -  “IMG_2922”

Cap 3 - “100_4768”

Cap 4 - “100_5648” (essa imagem em especial o foco não está perfeito, mas gostaria que entrasse assim mesmo)

Cap 5 -  “IMG_4928”

Cap 6 - “100_3892”

Cap 7 - “100_4966”

Cap 8 - “IMG_1553”

Cap 9 - “100_1752"



2- As outras fotos indicadas dentro do livro são, conforme está indicado no texto:

Prólogo - “Foto da Aldeia Awa”  - IMG_4983


1º Capítulo

Mapa

Foto aérea de Sebastião Salgado - Paisagem_SS (Sebastião Salgado)


2º Capítulo

Foto Domenico - Porcos - Domenico_V1A8262 (Domenico Pugliese)


3º Capítulo

Na p.40, no subtítulo “comida” entrará a foto do pessoal comendo mel. nome do arquivo - IMG_1672

Foto de ximixia - IMG_4613


6º Capítulo
Foto do quati sendo mexido - IMG_0343

Na p.23 estava escrito “foto do mel”, mas não vai entrar aqui pois já terá entrado no capítulo 3.


7º Capítulo
Mostrar foto de Pinawxiká com sua espingarda - IMG_4463

Kamaraxa’a alimentando as flechas - 100_5107


8º Capítulo
Foto caçadores olhando pro alto - 100_5208

Foto da volta da espera, hamo e juma’ã - IMG_1669 (mudei essa foto, e coloquei de um cara fazendo o arco, mas entra no mesmo lugar).

Na parte onde está escrito "foto da tocaia” eu mudei de imagem e vai entrar essa aqui - IMG_1615


9º Capítulo
Foto da takaja - 100_1695

========================================================

AJUSTE DE IMAGENS E TABELAS:

Além de...
- A tabela da página 373, na coluna do meio (afinidade), não puxou o símbolo que ele usou. Mesmo no original não consta, acho que talvez seja de uma fonte que se perdeu na passagem do word pro l.o.

- A tabela da página 574 ainda está com alguns negritos, porque tem muitos termos em itálico e não encontrei uma solução para substituir os negritos e não confundir com as expressões italizadas. Pensar se seria o caso de deixar tudo em redondo e manter itálico os termos que estavam em bold.

========================================================

PAGINAÇÕES CORRESPONDEM À VERSÃO d87f9d0

OK + Outra coisa, no prólogo, quando eu falo do número da população eu menciono “ver censo em anexo”, mas não vou colocar esse censo, vai ser demais. O livro só vai ficar maior e não acrescenta muito no fim. 
Teria só que tirar essa observação “ver censo em anexo”. Pode ser?

OK + Mas já digo de cara que naquela minha descrição sobre o autor eu queria que aparecesse “carioca” também, Tipo “Uirá Garcia, carioca, antropólogo…” acho que não tem problema né?

OK + E o meu nome de autor é "Uirá Garcia". Uirá Felippe Garcia é meu nome completo, mas como autor (em artigos, por exemplo) eu uso Uirá Garcia. Pode ser né?

OK + p.29. depois da frase “a despeito da revolução neolítica”, precisa entrar uma vírgula.

OK + p.378. (l. 1738) A narrativa está na página 377 mas a fonte que eu cito da narrativa está na página 378.

OK + Bibliografia: diagramar e verificar links. VER ALABAMA PRESS (p. 636). SAINDO SÍMBOLO NO PDF.

UIRÁ + PRETAS: melhorar texto do livro.

UIRÁ MANTIVE + p.74. Trocar o “o que” por “isso” na 5ª linha “canoas, isso representa...”

UIRÁ OK. MANTER BARRA??? + p.595. Sobre a passagem: “São também "comedores" (embora os Guajá não usem esse termo)...”. Ela está errada. EU ERREI. Temos que tirar esse parêntese “(embora os Guajá não usem esse termo)”. E colocar outro. O certo é: “São também "comedores" (i’uhara/ ou me’ehara), pois são "caçadores"...”

================================

+ p.216. O diagrama pode ficar um pouco menor. Tá muito grande.

+ p.287. Os diagramas no geral podem ficar menores.

OK + p.313. (l. 1519 CAP_5) Atenção: O Diagrama está errado! (veja que na página 346 o mesmo diagrama está lá. Só que lá está certo). O diagrama da página 313 é esse aqui abaixo:

+ p.373. Sobre as tabelas das página 373 e 374 são essas aqui. O sinal que está faltando é essa setinha de vai e vem.


+ p.449. (Cap. 7, l. 1430) foto que está nessa página deve ir para a página 455 ou 456, pois faz parte desse tópico.

+ p.570. TABELA: 
b) No item “não indígena (branco)”, na sessão karawara colocar, embaixo do nome jawarymỹ a descrição “cachorro morto”.

