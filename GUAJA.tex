
\chapter{Como ler as palavras guajá}

%Língua e convenção ortográfica

A língua Guajá foi inicialmente estudada por Péricles Cunha (1988) e
mais recentemente por Marina Magalhães (2007). Esta língua pertence ao
subgrupo \textsc{viii} da família linguística Tupi-Guarani, que inclui também o
Takunyapé, o Ka'apor, o Waiampi, o Wayampipukú, o Emérillon, o Amanayé,
o Anambé, o Turiwára e o Zo'é (Rodrigues, 1984/85 e Magalhães 2007). A
grafia dos termos em Guajá aqui utilizadas baseia"-se fundamentalmente no
trabalho de Magalhães (2007) que realiza uma análise morfológica e
sintática dessa língua. Todas as palavras na língua Guajá, bem como em
outras línguas indígenas, estão em itálico e as traduções entre aspas.
Utilizei uma convenção fonética cujos valores dos sons aproximados são:

\section{Vogais}

\begin{itemize}
\item[a] Vogal central baixa (como \emph{a} em português);

\item[e] Vogal anterior média não arredondada (como \emph{e} em português);

\item[i] Vogal anterior alta não arredondada (como \emph{i} em português);

\item[y] Vogal central alta não arredondada (tal como encontrada em outras línguas Tupi);

\item[o] Vogal posterior média arredondada (como \emph{o} em português);

\item[u] Vogal posterior alta arredondada (como \emph{u} em português);
\end{itemize} 


Todas as vogais podem ser nasalizadas; para tanto, utilizo o til em
todos os casos {[}ã, ẽ, ĩ, ỹ, õ, ũ{]}, e geralmente a presença de uma
vogal nasal acarreta a nasalização das vogais e consoantes que lhes são
próximas.

\section{Consoantes}

\begin{itemize}
\item[p] Oclusiva bilabial surda (como \emph{p} em português);

\item[t] Oclusiva alveolar surda (como \emph{t} em português);

\item[x] Oclusiva dental palatalizada (soa como o \emph{t} de \emph{tia} no falar carioca, porém ocorrendo diante das vogais);

\item[k] Oclusiva velar surda (como o \emph{c} em \emph{casa} em português);

\item[kw] Oclusiva velar surda labializada (como em \emph{quarto} em português);

\item[m] Nasal bilabial (como \emph{m} em português);

\item[n] Nasal dental (como \emph{n} em português);

\item[j] Aproximante palatal cujo som é o equivalente ao produzido em ditongos com \emph{i}, como nas palavras ``sai'' e ``meia'' em português; seguida de vogal nasal (ex.: \emph{jã, `cantar'}), passa a ser pronunciada como nasal alveopalatal sonora (como o \emph{nh} em \emph{manhã} em português)

\item[r] Vibrante simples (tepe) (como \emph{para} em português);

\item[h] Fricativa glotal (como \emph{heaven} em inglês);

\item[w] Contínua bilabial sem fricção (como em \emph{power} em inglês);

\item['] Oclusão glotal;
\end{itemize}

Assim como as vogais, muitas consoantes podem ser nasalizadas, e são
aqui identificadas com um til.

A ausência de acento na ortografia da língua deve"-se à previsibilidade
das sílabas tônicas já que, de maneira regular, todas as palavras são
oxítonas, isto é, terminam em sílabas tônicas, exceto as que terminam em
vogal \emph{a} depois de outra vogal ou depois das consoantes r, n e j.
Assim, \emph{awa} e \emph{mukuri} são pronunciadas como ``awá'' e
``mukurí'', mas \emph{mihua}, \emph{jakarea, Maira, amỹna e takaja} são
pronunciadas como ``mihúa'', ``jakaréa'', ``Maíra'', ``amỹna'' e
``takája''.

Quanto à grafia dos povos indígenas, acompanho a convenção da Associação
Brasileira de Antropologia, sem flexionar os nomes.


