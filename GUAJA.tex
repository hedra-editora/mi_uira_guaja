\chapter{Como ler as palavras guajá}

%Língua e convenção ortográfica

A língua guajá foi inicialmente estudada por Péricles Cunha em 1988 e
mais recentemente por Marina Magalhães, em 2007. Pertence ao
subgrupo \textsc{viii} da família linguística tupi-guarani, que inclui também o
takunyapé, o ka'apor, o waiampi, o wayampipukú, o emérillon, o amanayé,
o anambé, o turiwára e o zo'é. %(Rodrigues, 1984/85 e Magalhães 2007)
A grafia dos termos em guajá aqui utilizadas baseia-se fundamentalmente no
trabalho de Magalhães que realiza uma análise morfológica e
sintática dessa língua. Todas as palavras na língua guajá, bem como em
outras línguas indígenas, estão em itálico e as traduções entre aspas.
Utilizei uma convenção fonética cujos valores dos sons aproximados são:

\section{Vogais}

\begin{tabular}{rl}
/a/ & vogal central baixa, como \textit{a} em português\\
/e/ & vogal anterior média não arredondada, como \textit{e} em português\\
/i/ & vogal anterior alta não arredondada, como \textit{i} em português\\
/y/ & vogal central alta não arredondada, como encontrada em outras línguas tupi\\
/o/ & vogal posterior média arredondada, como \textit{o} em português\\
/u/ & vogal posterior alta arredondada, como \textit{u} em português
\end{tabular} 

\bigskip
\medskip

Todas as vogais podem ser nasalizadas; para tanto, utilizo o \textit{til} em
todos os casos --- \textit{ã}, \textit{ẽ}, \textit{ĩ}, \textit{ỹ}, \textit{õ}, \textit{ũ} ---, e geralmente a presença de uma
vogal nasal acarreta a nasalização das vogais e consoantes que lhes são
próximas.

\section{Consoantes}

\begin{tabular}{rl}
/p/ & oclusiva bilabial surda, como \textit{p} em português\\
/t/ & oclusiva alveolar surda, como \textit{t} em português\\
/x/ & oclusiva dental palatalizada, como \textit{t} de \textit{tia}, diante das vogais\\
/k/ & oclusiva velar surda, como \textit{c} em \textit{casa} em português\\
/kw/ & oclusiva velar surda labializada, como \textit{quarto} em português\\
/m/ & nasal bilabial, como \textit{m} em português\\
/n/ & nasal dental, como \textit{n} em português\\
/j/ & palatal equivalente aos ditongos com \textit{i}, como \textit{sai} ou \textit{meia}\footnote{Seguida de vogal nasal como \textit{jã} ou \textit{cantar}, passa a ser pronunciada como nasal alveopalatal sonora, como o \textit{nh} de \textit{manhã} em português.}\\
/r/ & vibrante simples, \textit{tepe}, como \textit{para} em português\\
/h/ & fricativa glotal, como \textit{heaven} em inglês\\
/w/ & contínua bilabial sem fricção, como em \textit{power} em inglês\\
/'/ & oclusão glotal
\end{tabular}

\bigskip
\medskip

Assim como as vogais, muitas consoantes podem ser nasalizadas, e são
aqui identificadas com um \textit{til}.

A ausência de acento na ortografia da língua deve-se à previsibilidade
das sílabas tônicas já que, de maneira regular, todas as palavras são
oxítonas. Isto é: terminam em sílabas tônicas --- exceto as finalizadas em
vogal \textit{a} depois de outra vogal, ou depois das consoantes \textit{r}, \textit{n} e \textit{j}.
Assim, \textit{awa} e \textit{mukuri} são pronunciadas como ``awá'' e
``mukurí'', mas \textit{mihua}, \textit{jakarea, Maira, amỹna} e \textit{takaja} são
pronunciadas como ``mihúa'', ``jakaréa'', ``Maíra'', ``amỹna'' e
``takája''.

Quanto à grafia dos povos indígenas, acompanho a convenção da Associação Brasileira de Antropologia, sem flexionar os nomes.


